\chapter{Programming practicals}
\label{sec:programming}

In this Chapter you will process some real geochronological datasets
using either \texttt{R} or \texttt{Matlab}. These are both
mathematical scripting languages that are both powerful and easy to
use.\\

\texttt{R} is a free and open programming language that works on any
operating system, including \texttt{Windows}, \texttt{OS-X} and
\texttt{Unix/Linux}. It can be downloaded and installed from
\texttt{http://r-project.org}.\\

\texttt{Matlab} is a proprietary programming environment. The full
version of this software is very expensive but a reasonably complete
student version can be purchased for $\pounds$55 + VAT from
\texttt{http://mathworks.com}. Alternatively, \texttt{Octave} is an
open source clone of \texttt{Matlab} that can be downloaded for free
from \texttt{http://www.gnu.org/software/octave}.\\

Sections~\ref{sec:R} and \ref{sec:Matlab} will present two brief
tutorials of \texttt{R} and \texttt{Matlab}, respectively.  These will
cover most commands that you will need for the subsequent computer
practicals.
