\section{Introduction to \texttt{Matlab}}
\label{sec:Matlab}

\begin{enumerate}
\item \texttt{Matlab}, like \texttt{R}, allows the user to run code
  either directly at the command line prompt, or from a built-in text
  editor. Anything following a percent symbol (\texttt{\%}) is a
  comment and is ignored. It is considered good practice to document
  your code with lots of comments, as will be done in this tutorial.
  
\begin{script}
% Ending a line with a semi-colon 
% suppresses output to the console:
'be loud'
'stay quiet';

% Matlab can be used as a fancy calculator:
1 + 1
sqrt(2)
exp(log(10))

% Intermediate values can be stored in variables:
foo = 2;
bar = 4;
foo = foo*bar;
foo
\end{script}

\item As the name suggest, \texttt{Matlab} (which stands for
  \texttt{MAT}rix \texttt{LAB}oratory) is capable of, and excels at,
  dealing with collections of multiple numbers arranged in vectors and
  matrices:

\begin{script}
% Data can be entered as vectors:
x = [0 2 4 6 8 10]
% or using shorthand syntax:
y = 0:2:10

% Define a matrix
z = [1 2 3
     4 5 6
     7 8 9];

% Accessing one or more elements 
% from a matrix or vector:
x(3)
y(1:3)
z(1:2,2:3)

% Attention: '*', '/' and '^' symbols
% are reserved for matrix operations:
a = [1 1 1]  % row vector
b = [1;1;1]  % column vector
a*b

% Use '.*', './' and '.^' for 
% element-wise operations:
a.*a

% Transpose of a matrix:
z'
\end{script}

\item In addition to Matlab's built-in operators and functions, you
  can also define your own:
  
\begin{script}
% take the cube of the input argument:
function out = cube(in)
  out = in^3;
end
\end{script}

Copy the \texttt{cube} function into a text file named \texttt{cube.m}
and save it in the current directory. You can then load the contents
of the file by simply typing its name without the \texttt{.m}
extension:

\begin{console}
> c3 = cube(3)
\end{console}

\item A useful feature of Matlab that is missing from most other
  programming languages is the ability to accommodate several output
  parameters:

\begin{script}
function [m,s] = basicStats(dat)
  m = mean(dat); % mean
  s = std(dat);  % standard deviation
end
\end{script}

Using the new function:

\begin{script}
% create a row vector of 10 random numbers
randnum = rand(1,10);

% use our new function
[m,s] = basicStats(randnum)
\end{script}

An extensive library of documentation is built into \texttt{Matlab}.
For example, to access further information about the \texttt{rand}
function, simply type \texttt{help rand} or \texttt{doc rand} at the
command prompt.

\item\texttt{Matlab} is a fully fledged programming language, which
  has flow structures such as \texttt{if} statements and \texttt{for}
  loops:

\begin{script}
% Conditional statement:
function coinToss
  % 'rand()' produces numbers between 0 and 1
  if (rand() > 0.5)
    % display a message to the console
    disp('head')
  else
    disp('tail')
  end
end
\end{script}

Use this function in a for loop:

\begin{script}
for i=1:10,
  coinToss
end
\end{script}

\item File and variable management:

\begin{script}
% List all the variables in the current workspace:
who
whos

% Remove some or all variables from the workspace:
clear m,s
who
clear all
who

% Basic file management is done with UNIX commands:
pwd   % pass the working directory
cd .. % move up one directory
ls    % list all files in the current directory
\end{script}

Use the above commands to navigate to the directory containing the
\texttt{RbSr.csv} file. To view the contents of this file:

\begin{console}
> type('RbSr.csv')
\end{console}

Incidentally, the \texttt{type} function can also be used to view the
code of Matlab functions:

\begin{console}
> type('mean')
\end{console}

\item Let us now apply our knowledge of \texttt{Matlab} to a
  geochronological problem:

\begin{script}
% Read the Rb-Sr dataset ignoring the header:
RbSr = csvread('RbSr.csv',1,0);

% Plot the Sr87Sr86-ratio (first column)
% against the Rb87Sr86-ratio (third column):
Rb87Sr86 = RbSr(:,1);
Sr87Sr86 = RbSr(:,3);
plot(Rb87Sr86,Sr87Sr86,'ok')
xlabel('87Rb/86Sr')
ylabel('87Sr/86Sr')
\end{script}

\item This should yield a linear array of Rb-Sr data. Let us now fit
  an isochron through these data. \texttt{Matlab} includes several
  functions to perform linear regression, including \texttt{regress}
  and \texttt{fitlm}. We will use the \texttt{polyfit} function:

\begin{script}
% Fit a first order polynomial through the data:
[fit,S] = polyfit(Rb87Sr86,Sr87Sr86,1);
% (S will be used in a later practical)

% define the 87Rb decay constant (in Ma-1):
lam87 = 1.42e-5;
% compute the age from the slope:
age = log(1 + fit(1))/lam87;

% predict the Sr87Sr86-ratio for the
% full range of Rb87Sr86-values:
x = [0,max(Rb87Sr86)];
y = fit(2) + fit(1)*x;

# add the prediction and age to the existing plot:
hold on;
plot(x,y,'-r')
title(age)
\end{script}

\end{enumerate}
