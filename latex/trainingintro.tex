\begin{refsection}
\chapter{Introduction}
\label{ch:intro1}

The use of naturally occurring radioactive isotopes to date minerals
and rocks is the oldest branch of isotope geology. The foundations of
these so-called isotopic or radiometric dating methods were laid
shortly after the turn of the XX\textsuperscript{th} century with the
discovery of the laws of radioactive decay by eminent physicists such
as Ernest Rutherford and Frederick Soddy \citep{rutherford1902a,
  rutherford1902b}.  The application of these principles to the field
of Geology and the calibration of the geological time scale were
pioneered by Arthur \citet{holmes1911, holmes1913,
  holmes1947}. Initially, radiometric geochronology was exclusively
based on uranium and its daughter products, but with the development
of increasingly sensitive analytical equipment, ever more isotopic
`clocks' were added over the course of the century: Rb/Sr
\citep{hahn1943}, $^{14}$C \citep{libby1946}, K/Ar
\citep{aldrich1948}, $^{238}$U fission tracks \citep{price1963},
$^{40}$Ar/$^{39}$Ar \citep{merrihue1966}, Sm/Nd \citep{lugmair1974},
etc.\\

The first part of these lecture notes provides a basic introduction to
all these methods. Chapter~\ref{ch:basic-notions} reviews the basic
principles of radioactive decay, which form the basis of all isotopic
dating techniques. It will derive the fundamental age equation and
introduce the concepts of secular equilibrium, which will be revisited
in later chapters. Chapter~\ref{ch:analyticaltechniques} provides the
briefest of introductions to the world of mass spectrometry. It will
sketch the basic operating principles of the instruments used to
acquire the datasets that will be used for \texttt{R} programming
exercises later on. Chapters~\ref{ch:intro2PD}--\ref{ch:intro2Useries}
provide basic introductions to the radiocarbon, Rb--Sr, Sm--Nd, U--Pb,
Ar--Ar, U--Th--He, fission track and Th--U methods, which will be
fleshed out further in Part~2 of the
notes.\\

Chapter~\ref{ch:error-propagation} presents a primer in error
propagation which is extremely important because, to quote K.R. Ludwig
``The uncertainty of the age is as important as the age
itself'' \citep{ludwig2003b}. Chapter~\ref{ch:exercises} contains a
collection of exercises that are meant to be solved by pencil on
paper, whereas Chapter~\ref{ch:programming} contains a collection of
practical exercises that require the \texttt{R} programming language.
In these exercises, you will process some raw data files for the
U--Pb, Ar--Ar and fission track methods. The purpose of these
exercises is to provide a glimpse into the `black box' data processing
software that is normally used by geochronologists to turn mass
spectrometer data into tables of isotopic ratios for further
processing with the \texttt{IsoplotR} package that is the subject of
Part~2 of this book.\\

The core of these notes is formed by Prof. Peter van den Haute's
lecture notes (in Dutch) at the University of Ghent.  This was
expanded with additional material, exercises, and practicals. Some
figures were modified from published sources,
including \citet{allegre2008}, \citet{braun2006},
and \citet{galbraith2005}. These books are recommended further reading
material, as is the detailed textbook by \citet{dickin2005}, from
which both \citet{allegre2008} and van den Haute heavily
borrowed. Additional lecture material, including the data files used
in the programming practicals of Chapter~\ref{ch:programming}, can be
found at \texttt{https://github.com/pvermees/geotopes/}.

\printbibliography[heading=subbibliography]
\end{refsection}
