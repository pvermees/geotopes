\begin{refsection}

\chapter{Generic functions}
\label{ch:generic-R}

\noindent\begin{minipage}[t]{.3\linewidth}
\strut\vspace*{-\baselineskip}\newline
\includegraphics[width=\linewidth]{../figures/OtherMethodsPlotdevices.png}\\
\end{minipage}
\begin{minipage}[t]{.7\textwidth}
  \texttt{IsoplotR} implements a number of plotting devices for 13
  different geochronometric methods. Some of these plotting devices
  are shared by multiple chronometers, and can also be used for
  datasets of non geochronological origin. This tutorial will start
  with an overview of these `generic' plots, whose user interface is
  shared by all the specific geochronological implementations that
  will be introduced subsequently.
\end{minipage}

In this Chapter, we will use a number of \texttt{IsoplotR}'s built-in
datasets, which can be downloaded from the \texttt{IsoplotR} GitHub
page, at
\url{https://github.com/pvermees/IsoplotR/tree/master/inst}. For the
tutorial in this Chapter, we will use the following datasets:

\begin{script}
data1 <- read.data('LudwigMixture.csv',method='other')
data2 <- read.data('RbSr1.csv',method='other')
data3 <- read.data('LudwigSpectrum.csv',method='other')
data4 <- read.data('LudwigMean.csv',method='other')
data5 <- read.data('LudwigKDE.csv',method='other')
\end{script}

Alternatively, the same files can also be found in the
\texttt{IsoplotR} installation folder on your computer, using
\texttt{R}'s built-in \texttt{system.file()} function. For example:

\begin{script}
fn <- system.file('LudwigMixture.csv',package='IsoplotR')
data1 <- read.data(fn,method='other')
\end{script}

\section{Radial plots}\label{sec:OtherRadial}

\noindent\begin{minipage}[t]{.27\linewidth}
  \strut\vspace*{-\baselineskip}\newline
  \includegraphics[width=\linewidth]{../figures/OtherRadialInput.png}
\end{minipage}
\begin{minipage}[t]{.73\linewidth}
  Radial plots require a table of measurements and their
  uncertainties, plus an optional vector with data that can be used to
  form a colour scale, and an optional list of data points to omit
  from the calculations or hide from the plot (see
  Sections~\ref{sec:GUI} and \ref{sec:CLI}).
\end{minipage}

\begin{console}
radialplot(data1)
\end{console}

The standard radial plot can be modified using a range of optional
arguments, which can be accessed from the GUI and CLI.

\begin{enumerate}

\item Labelling the error ellipses can be useful to identify outliers
  or otherwise noteworthy aliquots.

  \noindent\begin{minipage}[t]{.28\linewidth}
  \strut\vspace*{-\baselineskip}\newline
  \includegraphics[width=\linewidth]{../figures/concordiashownumbers.png}
\end{minipage}
\begin{minipage}[t]{.72\linewidth}
  Ticking the box replaces the plot symbols with the row numbers of
  the input data.
\end{minipage}

\begin{console}
radialplot(data1,show.numbers=TRUE)
\end{console}
  
\item \texttt{IsoplotR} offers three types of transformations for the
  radial scale.
  
\noindent\begin{minipage}[t]{.3\linewidth}
  \strut\vspace*{-\baselineskip}\newline
  \includegraphics[width=\linewidth]{../figures/UPbRadialTransformation.png}
\end{minipage}
\begin{minipage}[t]{.7\linewidth}
The logarithmic transformation works well for geochronological data,
which are constrained to strictly positive numbers. The square root
transformation is advisable for datasets that contain very small
numbers compared to the analytical uncertainty.  The linear scale is
appropriate for quantities that are free to range from $-\infty$ to
$+\infty$.
\end{minipage}

\begin{console}
radialplot(data1,transformation='sqrt')
\end{console}

\item All of \texttt{IsoplotR}'s mixture models functionality is
  covered under its radial plot function, including the discrete
  mixture and minimum age models:

\noindent\begin{minipage}[t]{.3\linewidth}
\strut\vspace*{-\baselineskip}\newline
\includegraphics[width=\linewidth]{../figures/UPbRadialMinagemod.png}
\end{minipage}
\begin{minipage}[t]{.7\linewidth}
  Selecting the numbers \texttt{1} through \texttt{5} applies the
  discrete mixture models of Equation~\ref{eq:mixture}. Selecting
  \texttt{auto} applies the Bayes Information Criterion to choose the
  `optimal' number of components, with the caveat of
  Figure~\ref{fig:increasingn}. Finally, selecting \texttt{minimum}
  applies the minimum age model of Equation~\ref{eq:Lminagemod}.
\end{minipage}

\begin{console}
radialplot(data1,k='min')
\end{console}

\item By default the minimum and maximum extent of the radial plot
  correspond to the minimum and maximum value in the dataset.

\noindent\begin{minipage}[t]{.45\linewidth}
\strut\vspace*{-\baselineskip}\newline
\includegraphics[width=\linewidth]{../figures/OtherRadialLimits.png}
\end{minipage}
\begin{minipage}[t]{.55\linewidth}
  These values can be overruled by entering the preferred values in
  the corresponding textboxes. A third box is available to specify the
  central value of the radial scale ($z_\circ$ in
  Equation~\ref{eq:radial}).
\end{minipage}

\begin{console}
radialplot(data1,from=10,to=200,z0=50)
\end{console}

\item Samples are represented by filled circles by default, but this
  can be changed to a range of other shape. These can be specified as
  numbers (\texttt{1-25}, see \texttt{?pch} for further details) or a
  single character such as \verb|'o'|, \verb|'*'|, \verb|'+'|, or
  \verb|'.'|. 

\noindent\begin{minipage}[t]{.5\linewidth}
\strut\vspace*{-\baselineskip}\newline
\includegraphics[width=\linewidth]{../figures/OtherRadialPCH.png}
\end{minipage}
\begin{minipage}[t]{.5\linewidth}
Using character~23 replaces the filled circles with filled diamonds.
\end{minipage}

\begin{console}
radialplot(data1,pch=23)
\end{console}
  
\item By default, \texttt{IsoplotR} reports confidence intervals (for
  the central age) on a 95\% confidence level.

\noindent\begin{minipage}[t]{.5\linewidth}
\strut\vspace*{-\baselineskip}\newline
\includegraphics[width=\linewidth]{../figures/OtherRadialAlpha.png}
\end{minipage}
\begin{minipage}[t]{.5\linewidth}
The significance level can be adjusted here, for example to a 99\%
confidence level.
\end{minipage}

\begin{console}
radialplot(data1,alpha=0.01)
\end{console}

\item The number of significant digits can be specified relative to
  the analytical uncertainty. For example, if \texttt{sigdig}=2, then
  $123.45678 \pm 0.12345$ is rounded to $123.46 \pm 0.12$.

\noindent\begin{minipage}[t]{.45\linewidth}
\strut\vspace*{-\baselineskip}\newline
\includegraphics[width=\linewidth]{../figures/OtherRadialSigdig.png}
\end{minipage}
\begin{minipage}[t]{.55\linewidth}
  The significant digits affect all numerical output that is reported
  in the figure legend.
\end{minipage}

\begin{console}
radialplot(data1,sigdig=3)
\end{console}

\item For plot characters \texttt{21-25}, the fill colour can be
  modified.

\noindent\begin{minipage}[t]{.5\linewidth}
\strut\vspace*{-\baselineskip}\newline
\includegraphics[width=\linewidth]{../figures/OtherRadialBG.png}
\end{minipage}
\begin{minipage}[t]{.5\linewidth}
One can either assign a fixed colour to all aliquots.
\end{minipage}

\begin{console}
radialplot(data1,bg='blue')
\end{console}

Alternatively, the fill colour can be used to visualise an additional
variable, by pasting a vector of values into the optional \texttt{(C)}
column of the input table (Section~\ref{sec:GUI}).

\noindent\begin{minipage}[t]{.5\linewidth}
\strut\vspace*{-\baselineskip}\newline
\includegraphics[width=\linewidth]{../figures/OtherRadialBGC.png}
\end{minipage}
\begin{minipage}[t]{.5\linewidth}
  The corresponding fill and stroke colours can be specified by
  supplying a vector of colours in the text box of the GUI.\\
\end{minipage}

For the sake of illustration, let us demonstrate this feature by
constructing a colour scale based on the relative measurement
uncertainties (uncertainty divided by age) of the test data:

\begin{script}
relerr <- data1[,2]/data1[,2]
radialplot(data1,levels=relerr,bg=c('white','red'))
\end{script}

\noindent\begin{minipage}[t]{.5\linewidth}
\strut\vspace*{-\baselineskip}\newline
\includegraphics[width=\linewidth]{../figures/OtherRadialClabel.png}
\end{minipage}
\begin{minipage}[t]{.5\linewidth}
  The colour scale can be labelled for clarity.
\end{minipage}

\begin{script}[firstnumber=2]
radialplot(data1,levels=relerr,bg=c('white','red'),clabel='s[t]/t')
\end{script}

\item The font size of the sample numbers, axis labels, and any legend
  can be adjusted with a multiplier.

\noindent\begin{minipage}[t]{.5\linewidth}
\strut\vspace*{-\baselineskip}\newline
\includegraphics[width=\linewidth]{../figures/OtherRadialCEX.png}
\end{minipage}
\begin{minipage}[t]{.5\linewidth}
Values greater than 1 increase the font size, values less than 1
reduce it.  
\end{minipage}

At the CLI, the font size is controlled by the environment variable
\texttt{cex}, which can be changed with the \texttt{par()} function:

\begin{script}
oldpar <- par(cex=0.9)
radialplot(data1)
par(oldpar) # restore the old cex value
\end{script}

\noindent\begin{minipage}[t]{.5\linewidth}
\strut\vspace*{-\baselineskip}\newline
\includegraphics[width=\linewidth]{../figures/OtherRadialPCHcex.png}
\end{minipage}
\begin{minipage}[t]{.5\linewidth}
The plot symbols are sized independently.
\end{minipage}

\begin{console}
radialplot(data1,cex=1.5)
\end{console}

\end{enumerate}

\section{Regression}\label{sec:OtherRegression}

\citet{york2004} regression requires a table of measurements for the
independent variable (\texttt{X}), the dependent variable
(\texttt{Y}), their respective uncertainties (\texttt{err[X]} and
\texttt{err[Y]}), and their correlation coefficient
(\texttt{err[rXY]}).\\

\noindent\begin{minipage}[t]{.4\linewidth}
  \strut\vspace*{-\baselineskip}\newline
  \includegraphics[width=\linewidth]{../figures/OtherRegressionInput.png}
\end{minipage}
\begin{minipage}[t]{.6\linewidth}
  Alternatively, the data can also be supplied as redundant ratios,
  which allow the error correlation to be computed from the
  uncertainties using Equation~\ref{eq:redundantratios}.
\end{minipage}

\begin{console}
isochron(data2)
\end{console}

An example using the second input option:

\begin{script}
d <- read.data('PbPb3.csv',method='other')
y <- data2york(d,format=3)
york(y)
\end{script}

\begin{enumerate}

\item \texttt{IsoplotR} offers three options to deal with the scatter of the
  data around the best-fit isochron line.

\noindent\begin{minipage}[t]{.45\linewidth}
\strut\vspace*{-\baselineskip}\newline
\includegraphics[width=\linewidth]{../figures/OtherRegressionModels.png}
\end{minipage}
\begin{minipage}[t]{.55\linewidth}
  These three models represent three different ways to capture any
  excess dispersion of the data relative to the nominal uncertainties
  (Figure~\ref{fig:isochronMSWD}).
\end{minipage}

\begin{console}
isochron(data2,model=3)
\end{console}
  
\item Just like the row numbers of the input data could be shown on a
  radial plots, so can they be added in a regression context.

  \noindent\begin{minipage}[t]{.28\linewidth}
  \strut\vspace*{-\baselineskip}\newline
  \includegraphics[width=\linewidth]{../figures/concordiashownumbers.png}
\end{minipage}
\begin{minipage}[t]{.72\linewidth}
Ticking the box adds numbers to the error ellipses.
\end{minipage}

\begin{console}
isochron(data2,show.numbers=TRUE)
\end{console}

\item The limits of the horizontal and vertical axis can be adjusted
  to any value.

\noindent\begin{minipage}[t]{.45\linewidth}
  \strut\vspace*{-\baselineskip}\newline
  \includegraphics[width=\linewidth]{../figures/OtherRegressionXYlims.png}
\end{minipage}
\begin{minipage}[t]{.55\linewidth}
Showing the full extent of the regression line, including the data and
the origin.
\end{minipage}

Adjusting the x and y-limits of a model-2 regression plot:

\begin{console}
isochron(data2,model=2,xlim=c(0,300),ylim=c(0,1500))
\end{console}

\item Other options are similar to the radial plot
  (Section~\ref{sec:OtherRadial}).

\noindent\begin{minipage}[t]{.45\linewidth}
  \strut\vspace*{-\baselineskip}\newline
  \includegraphics[width=\linewidth]{../figures/OtherRegressionAlpha.png}
\end{minipage}
\begin{minipage}[t]{.55\linewidth}
  The default values for the significance level (\texttt{alpha}=0.05),
  number of significant digits (\texttt{sigdig}=2), and font
  magnification (\texttt{cex}=1) can be changed by entering the
  appropriate value in these text boxes.
\end{minipage}

\begin{script}
oldpar <- par(cex=1.1)
isochron(data2,alpha=0.01,sigdig=4)
par(oldpar)
\end{script}

\item Finally, the fill and stroke colour of the error ellipses can be
  modified and an optional colour scale added to the plot

\noindent\begin{minipage}[t]{.45\linewidth}
  \strut\vspace*{-\baselineskip}\newline
  \includegraphics[width=\linewidth]{../figures/OtherRegressionFillStroke.png}
\end{minipage}
\begin{minipage}[t]{.55\linewidth}
For the sake of illustration we here use a colour ramp to fill the
ellipses according to the error correlation, and paint the outline of
the ellipses a uniform red.
\end{minipage}

\begin{script}
isochron(data2,levels=data2[,'rXY'],clabel='rho',
         ellipse.fill=topo.colors(n=100),ellipse.stroke='red')
\end{script}
  
\end{enumerate}

\section{Weighted means}\label{sec:OtherWeightedMean}

\noindent\begin{minipage}[t]{.35\linewidth}
  \strut\vspace*{-\baselineskip}\newline
  \includegraphics[width=\linewidth]{../figures/OtherWtdMeanData.png}
\end{minipage}
\begin{minipage}[t]{.65\linewidth}
\end{minipage}

\begin{enumerate}

\item \noindent\begin{minipage}[t]{.3\linewidth}
  \strut\vspace*{-\baselineskip}\newline
  \includegraphics[width=\linewidth]{../figures/OtherWtdMeanRandomEffects.png}
\end{minipage}
\begin{minipage}[t]{.7\linewidth}
\end{minipage}

\item \noindent\begin{minipage}[t]{.21\linewidth}
  \strut\vspace*{-\baselineskip}\newline
  \includegraphics[width=\linewidth]{../figures/OtherWtdMeanOutliers.png}
\end{minipage}
\begin{minipage}[t]{.79\linewidth}
\end{minipage}

\item \noindent\begin{minipage}[t]{.21\linewidth}
  \strut\vspace*{-\baselineskip}\newline
  \includegraphics[width=\linewidth]{../figures/OtherWtdMeanRank.png}
\end{minipage}
\begin{minipage}[t]{.79\linewidth}
\end{minipage}

\item \noindent\begin{minipage}[t]{.45\linewidth}
  \strut\vspace*{-\baselineskip}\newline
  \includegraphics[width=\linewidth]{../figures/OtherWtdMeanRemainingOptions.png}
\end{minipage}
\begin{minipage}[t]{.55\linewidth}
\end{minipage}

\end{enumerate}
  
\section{Age spectra}\label{sec:OtherAgeSpectra}

\section{Kernel density estimates}\label{sec:OtherKDE}

\section{Cumulative age distributions}\label{sec:OtherCAD}

\printbibliography[heading=subbibliography]

\end{refsection}
