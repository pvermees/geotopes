\begin{refsection}

\chapter{Extending \texttt{IsoplotR}}\label{ch:extending-IsoplotR}

\texttt{IsoplotR} aims to cover the most commonly used functions in
geochronology. However geochronology is such a diverse and rapidly
evolving field of science, that it is impossible for one piece of
software to cover everything. This is why, from its inception,
\texttt{IsoplotR} was designed with extendability in
mind. Section~\ref{sec:API} gives an overview of the code base for the
CLI, which can be used as building blocks for automation scripts,
alternative visualisations and new
applications. Section~\ref{sec:schema} summarises the \texttt{.json}
schema that is used to import and export data to the GUI. Using this
schema, it is possible to connect lower level data reduction software
(written in any programming language) to \texttt{IsoplotRgui}.

\section{List of functions}\label{sec:API}

This Section gives a brief summary of \texttt{IsoplotR}'s Application
Programming Interface (API). This is a list of all \texttt{IsoplotR}'s
public functions. To view this list within \texttt{R}, enter:

\begin{console}
help(package='IsoplotR')
\end{console}

\noindent at the CLI. The number of public functions was intentionally
kept small, so as to shorten the learning curve. However despite this
apparent simplicity, \texttt{IsoplotR}'s code base offers a lot of
flexibility. This is because:

\begin{enumerate}
\item several functions serve multiple purposes. For example, the
  \texttt{settings()} function can be used to get or set decay
  constants, isotopic ratios and other global parameters, and the
  \texttt{peakfit()} function groups methods to compute finite
  mixtures as well as minimum age models.
\item \texttt{IsoplotR} is built around 13 so-called S3 classes,
  which are used to store different types of chronometric data. For
  example, a single function called \texttt{age()} can be used to
  calculate U--Pb, Th--Pb, Pb--Pb, Ar--Ar, Rb--Sr, Sm--Nd, Lu--Hf,
  Re--Os, fission track, U--Th--He or Th--U data, both from datasets
  of multiple aliquots, or for individual measurements.
\item some functions can be used to both compute and plot data.  For
  example, the \texttt{kde()} function generates kernel density plots,
  whilst returning the plot coordinates and optimal bandwidth to the
  user. It is possible to suppress the graphical output but retain the
  numerical output. Thus, the \texttt{kde()} function can be used to
  write one's own visualisation. The same is true for the
  \texttt{isochron()} and \texttt{weightedmean()} functions.
\item \texttt{IsoplotR} is entirely written in \texttt{R} and does not
  depend on any non-standard packages. This makes the package small
  and easy to install. Which means that, if your \texttt{R} code
  requires uses one of \texttt{IsoplotR}'s functions, then this won't
  bloat your program. It also means that it is relatively
  straightforward to lift functions out of \texttt{IsoplotR} and copy
  them into another program. You have the permission to do so as long
  as the origin of the code is documented in the new program, and your
  program is released under the GPL-3 license, like \texttt{IsoplotR}
  itself.
\end{enumerate}

Detailed documentation can be obtained from within \texttt{R}, using
the \texttt{help} or \texttt{?} functions. For example, to view
the documentation of the \texttt{isochron()} function, type

\begin{console}
?isochron
\end{console}

\noindent at the command prompt. The detailed documentation covers
well over 100 pages and will not be reproduced here. Instead we
suffice with a simple list of the functions accompanied by a brief
summary of their input and output:

\begin{console}

age(x,method,exterr,J,zeta,rhoD,d,...)
            
\end{console}

age                     Calculate isotopic ages
age2ratio               Predict isotopic ratios from ages
agespectrum             Plot a (40Ar/39Ar) release spectrum
cad                     Plot continuous data as cumulative age
                        distributions
central                 Calculate U-Th-He and fission track central
                        ages and compositions
classes                 Geochronological data classes
concordia               Concordia diagram
data2york               Prepare geochronological data for York
                        regression
discfilter              Set up a discordance filter
diseq                   Set up U-series disequilibrium correction for
                        U-Pb geochronology
ellipse                 Get error ellipse coordinates for plotting
evolution               Th-U evolution diagram
examples                Example datasets for testing 'IsoplotR'
helioplot               Visualise U-Th-He data on a logratio plot or
                        ternary diagram
isochron                Calculate and plot isochrons
kde                     Create (a) kernel density estimate(s)
ludwig                  Linear regression of U-Pb data with correlated
                        errors, taking into account decay constant
                        uncertainties.
mclean                  Predict disequilibrium concordia compositions
mds                     Multidimensional Scaling
Pb0corr                 Common Pb correction
peakfit                 Finite mixture modelling of geochronological
                        datasets
radialplot              Visualise heteroscedastic data on a radial plot
read.data               Read geochronology data
scatterplot             Create a scatter plot with error ellipses or
                        crosses
set.zeta                Calculate the zeta calibration coefficient for
                        fission track dating
settings                Load settings to and from json
titterington            Linear regression of X,Y,Z-variables with
                        correlated errors
weightedmean            Calculate the weighted mean age
york                    Linear regression of X,Y-variables with
                        correlated errors


  \section{\texttt{json} schema}\label{sec:schema}

\printbibliography[heading=subbibliography]

\end{refsection}
