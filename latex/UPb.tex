\begin{refsection}

\chapter{U--Pb geochronology}\label{ch:U-Pb}

\section{Input formats}
\label{sec:UPbFormats}

\texttt{IsoplotR} offers eight different input formats:
\begin{enumerate}
  \item
  $\frac{07}{35}$,  
  $\mbox{err}\!\left[\frac{07}{35}\right]$, 
  $\frac{06}{38}$,  
  $\mbox{err}\!\left[\frac{06}{38}\right]$,  
  $\mbox{r}\!\left[\frac{06}{38},\frac{06}{38}\right]$
  \item $\frac{38}{06}$,  
  $\mbox{err}\!\left[\frac{38}{06}\right]$, 
  $\frac{07}{06}$,  
  $\mbox{err}\!\left[\frac{07}{06}\right]$,  
  $\left(\mbox{r}\!\left[\frac{38}{06},\frac{07}{06}\right]\right)$
  \item
  $\frac{07}{35}$,  
  $\mbox{err}\!\left[\frac{07}{35}\right]$, 
  $\frac{06}{38}$,  
  $\mbox{err}\!\left[\frac{06}{38}\right]$, 
  $\frac{07}{06}$,  
  $\mbox{err}\!\left[\frac{07}{06}\right]$, 
  $\left(\mbox{r}\!\left[\frac{07}{35},\frac{06}{38}\right]\right)$,  
  $\left(\mbox{r}\!\left[\frac{07}{35},\frac{07}{06}\right]\right)$, 
  $\left(\mbox{r}\!\left[\frac{06}{38},\frac{07}{06}\right]\right)$
  \item 
  $\frac{07}{35}$,  
  $\mbox{err}\!\left[\frac{07}{35}\right]$, 
  $\frac{06}{38}$,  
  $\mbox{err}\!\left[\frac{06}{38}\right]$,  
  $\frac{04}{38}$,  
  $\mbox{err}\!\left[\frac{04}{38}\right]$, 
  $\left(\mbox{r}\!\left[\frac{07}{35},\frac{06}{38}\right]\right)$,  
  $\left(\mbox{r}\!\left[\frac{07}{35},\frac{04}{38}\right]\right)$, 
  $\left(\mbox{r}\!\left[\frac{06}{38},\frac{04}{38}\right]\right)$
  \item 
  $\frac{38}{06}$,  
  $\mbox{err}\!\left[\frac{38}{06}\right]$, 
  $\frac{07}{06}$,  
  $\mbox{err}\!\left[\frac{07}{06}\right]$,  
  $\frac{04}{06}$,  
  $\mbox{err}\!\left[\frac{04}{06}\right]$, 
  $\left(\mbox{r}\!\left[\frac{38}{06},\frac{07}{06}\right]\right)$,  
  $\left(\mbox{r}\!\left[\frac{38}{06},\frac{04}{06}\right]\right)$, 
  $\left(\mbox{r}\!\left[\frac{07}{06},\frac{04}{06}\right]\right)$
  \item 
  $\frac{07}{35}$,  
  $\mbox{err}\!\left[\frac{07}{35}\right]$, 
  $\frac{06}{38}$,  
  $\mbox{err}\!\left[\frac{06}{38}\right]$,  
  $\frac{04}{38}$,  
  $\mbox{err}\!\left[\frac{04}{38}\right]$,  
  $\frac{07}{06}$,  
  $\mbox{err}\!\left[\frac{07}{06}\right]$, 
  $\frac{04}{07}$,  
  $\mbox{err}\!\left[\frac{04}{07}\right]$,  
  $\frac{04}{06}$,  
  $\mbox{err}\!\left[\frac{04}{06}\right]$
  \item 
  $\frac{07}{35}$,  
  $\mbox{err}\!\left[\frac{07}{35}\right]$, 
  $\frac{06}{38}$,  
  $\mbox{err}\!\left[\frac{06}{38}\right]$,  
  $\frac{08}{32}$,  
  $\mbox{err}\!\left[\frac{08}{32}\right]$,  
  $\frac{32}{38}$,  
  $\mbox{err}\!\left[\frac{32}{38}\right]$,  \\
  $\left(\mbox{r}\!\left[\frac{07}{35},\frac{06}{38}\right]\right)$,  
  $\left(\mbox{r}\!\left[\frac{07}{35},\frac{08}{32}\right]\right)$, 
  $\left(\mbox{r}\!\left[\frac{07}{35},\frac{32}{38}\right]\right)$,  
  $\left(\mbox{r}\!\left[\frac{06}{38},\frac{08}{32}\right]\right)$,  
  $\left(\mbox{r}\!\left[\frac{06}{38},\frac{32}{38}\right]\right)$, 
  $\left(\mbox{r}\!\left[\frac{08}{32},\frac{32}{38}\right]\right)$
  \item
  $\frac{38}{06}$,  
  $\mbox{err}\!\left[\frac{38}{06}\right]$, 
  $\frac{07}{06}$,  
  $\mbox{err}\!\left[\frac{07}{06}\right]$,  
  $\frac{08}{06}$,  
  $\mbox{err}\!\left[\frac{08}{06}\right]$,  
  $\frac{32}{38}$,  
  $\mbox{err}\!\left[\frac{32}{38}\right]$,  \\
  $\left(\mbox{r}\!\left[\frac{38}{06},\frac{07}{06}\right]\right)$,  
  $\left(\mbox{r}\!\left[\frac{38}{06},\frac{08}{06}\right]\right)$, 
  $\left(\mbox{r}\!\left[\frac{38}{06},\frac{32}{38}\right]\right)$,  
  $\left(\mbox{r}\!\left[\frac{07}{06},\frac{08}{06}\right]\right)$,  
  $\left(\mbox{r}\!\left[\frac{07}{06},\frac{32}{38}\right]\right)$, 
  $\left(\mbox{r}\!\left[\frac{08}{06},\frac{32}{38}\right]\right)$
\end{enumerate}

\noindent where 04, 06, 07, 08, 32, 35 and 38 stand for
\textsuperscript{204}Pb, \textsuperscript{206}Pb,
\textsuperscript{207}Pb, \textsuperscript{208}Pb,
\textsuperscript{232}Th, \textsuperscript{235}U and
\textsuperscript{238}U, respectively. `err[$\ast$]' stands for the
analytical uncertainty of $\ast$, which can be specified as a standard
error or as two times the standard error, either in absolute or
relative units. And `r[$x$,$y$]' stands for the error correlation
between $x$ and $y$.\\

Formats~1--3 are meant for mass spectrometers that are unable to
accurately measure \textsuperscript{204}Pb. This is the case for
single collector ICP-MS instruments that are unable to resolve the
isobaric interference on \textsuperscript{204}Hg, which is often
present in the plasma gas. Formats~4--6 include
\textsuperscript{204}Pb, as measured by SIMS, TIMS or multi-collector
ICP-MS. Finally, formats~7 and 8 include \textsuperscript{208}Pb and
\textsuperscript{232}Th. These nuclides can be used for hybrid
U--Th--Pb dating, as discussed in Section~\ref{sec:U-Th-Pb}.\\

Formats~1, 4 and 7 are `Wetherill style' input formats, in which the
radioactive parent appears in the denominator of the isotopic ratio
data. As explained in Section~\ref{sec:errorcorrelations} and shown in
Figure~\ref{fig:errorcorrelation}, these formats are associated with
strong error correlations
($r\left[\frac{07}{35},\frac{06}{38}\right]$), which must be specified
so as to avoid inaccurate inferences.  Formats~2, 5 and 8 are
`Tera-Wasserburg style' input formats, in which the most abundant
radiogenic daughter (i.e., \textsuperscript{206}Pb) appears in the
denominator of the isotopic ratio data.  As shown in
Figure~\ref{fig:inverrorcorrelation}, this greatly reduces the error
correlations which, consequently, are optional (hence the brackets
around $r\left[\frac{38}{06},\frac{07}{06}\right]$).\\

Finally, formats~3 and 6 provide an alternative input format designed
for users whose low level data processing software does not provide
error correlation data. It uses redundant ratios to infer the
correlation coefficients. Let $X \equiv \frac{07}{35}$, $Y \equiv
\frac{07}{35}$, $Z \equiv \frac{07}{06}$ and $U \equiv \frac{38}{35}$,
let $s[X]$, $s[Y]$ and $s[Z]$ be the standard errors of $X$, $Y$ and
$Z$, and assume that $s[U]=0$ for the sake of simplicity. Then it is
easy to see that $Z = X/(U Y)$, and
\begin{equation}
  \left(\frac{s[Z]}{Z}\right)^2 = \left(\frac{s[X/Y]}{X/Y}\right)^2
  \approx \left(\frac{s[X]}{X}\right)^2 + \left(\frac{s[Y]}{Y}\right)^2 -
  2 \frac{s[X,Y]}{XY}
\end{equation}

\noindent from which the covariance (and, hence, the correlation
coefficient) between $X$ and $Y$ can be inferred as
\begin{equation}
  s[X,Y] \approx \frac{XY}{2}
  \left[
    \left(\frac{s[X]}{X}\right)^2 +
    \left(\frac{s[Y]}{Y}\right)^2 -
    \left(\frac{s[Z]}{Z}\right)^2
    \right]
\end{equation}

It is important to note that this approach makes the crucial
assumption that all three standard errors ($s[X]$, $s[Y]$ and $s[Z]$)
are based on the same number of data points. This means that formats~3
and 6 are not applicable to TIMS data.

\section{Concordia diagrams and ages}
\label{sec:concordia}

\printbibliography[heading=subbibliography]

\end{refsection}
