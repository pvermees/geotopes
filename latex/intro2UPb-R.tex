\section{U-Th-Pb data reduction}
\label{sec:U-Pb-R}

You are supplied with two data files that were produced by the
quadrupole laser ablation ICP-MS system at UCL's London Geochronology
Centre. At the time of the analysis, this instrument could not resolve
$^{204}$Pb from the isobaric interference at $^{204}$Hg. Therefore, it
is not possible to apply a common lead correction as explained in
Section \ref{sec:U-Pb}. However, this does not cause any major issues
to us because:

\begin{enumerate}
\item The mineral analysed is zircon, which incorporates very little
  common Pb in its crystal structure during crystallisation.
\item The ages are sufficiently old for the radiogenic Pb to dominate
  the common Pb component by orders of magnitude.
\end{enumerate}

In this exercise, we will use standard-sample bracketing
(Section~\ref{sec:bracketing}) to process some raw mass spectrometer
data in \texttt{R} \ifuclnotes or \texttt{Matlab}\fi:

\begin{enumerate}
\item Load the input files {\tt 91500.csv} (sample) and {\tt GJ1.csv}
  (standard) into memory.
\item Plot the $^{238}$U signal against time. The resulting curve
  consists of three segments: (i) the first 20 seconds record the
  background (`blank') signal of the ICP-MS, measured while the laser
  was switched off; (ii) 20~seconds into the run, the laser is turned
  on and the ions arrive in the ICP-MS; (iii) After the signal has
  ramped up quickly, it slowly drops over time as the laser goes out
  of focus whilst drilling deeper into the sample. This is the
  `signal'.
\item Compute the arithmetic mean U and Pb blank (measurements before
  20 seconds), and subtract them from the signal (measurements after
  25 seconds). Do this for both the sample and the standard.  You will
  get two times four vectors, for $^{206}$Pb, $^{207}$Pb, $^{235}$U
  and $^{238}$U.
\item Use the four blank corrected signal vectors to form two pairs of
  $^{206}$Pb/$^{238}$U and $^{207}$Pb/$^{235}$U vectors.
\item Take the arithmetic mean of the $^{206}$Pb/$^{238}$U and
  $^{207}$Pb/$^{235}$U ratio vectors. You will now have two pairs of
  numbers representing the \emph{measured} $^{206}$Pb/$^{238}$U and
  $^{207}$Pb/$^{235}$U ratios for the sample and the standard.
\item Given that GJ-1 has a known age of 600.4 Ma, what are its
  \emph{expected} $^{206}$Pb/$^{238}$U, and $^{207}$Pb/$^{235}$U
  ratios? Is there a significant difference between the measured and
  the expected ratios? What could be causing this?
\item\label{it:corr} Calculate a correction factor by dividing the
  expected GJ-1 ratios by the measured values.
\item\label{it:atomicUPb} Apply the correction factor calculated in
  step~\ref{it:corr} to the measured ratios for sample 91500. This
  gives us two estimated atomic $^{206}$Pb/$^{238}$U and
  $^{207}$Pb/$^{235}$U ratios.
\item What is the age of 91500?
\item Can you plot the results on a Wetherill concordia diagram?
\end{enumerate}
