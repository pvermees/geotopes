\section{Fission tracks}
\label{sec:FT-R}

In this exercise, you will use your programming skills to calculate
some fission track ages. You are given the following datasets:

\begin{enumerate}
\item\texttt{DUR.csv}: a table with two columns listing the number of
  spontaneous tracks $N_s$ and induced tracks $N_i$ counted in 25
  grains of an apatite age standard (t = 31.4 Ma) from Durango,
  Mexico.  Note that these pairs of tracks were counted over the same
  area, so that $\rho_s/\rho_i$ = $N_s/N_i$ in
  Equation~\ref{eq:tzeta}.
\item\texttt{MD.csv}: a similar table for an apatite sample from Mount
  Dromedary, Australia.
\end{enumerate}

You will need to:

\begin{enumerate} 

\item Rewrite Equation \ref{eq:tzeta} in terms of the $\zeta$
  calibration factor and use this new formula to calculate the $\zeta$
  factor for each single grain analysis of the Durango age standard.
  Use a dosimeter track density of $\rho_D = 300,000$ cm$^{-2}$.
\item Use the mean of these $\zeta$ factors to calculate the age of
  the Mount Dromedary sample (i.e., the single grain ages and their
  mean).
\item Propagate the analytical uncertainties for each of those single
  grain ages, using the fact that fission track counts (N, say) follow
  a Poisson distribution for which it is true that:

$$\sigma^2_N = N$$

To simplify the calculations, you can also use the following
approximation:

$$\frac{1}{\lambda}\ln\left(1+\frac{g_i}{g_s}\lambda\zeta\rho_d\frac{N_s}{N_i}\right)
\approx \frac{g_i}{g_s}\zeta\rho_d\frac{N_s}{N_i}$$

\item How does the single grain age precision of the fission track
  method compare to the U-Pb and $^{40}$Ar/$^{39}$Ar age uncertainties
  in Sections \ref{sec:U-Pb-R} and \ref{sec:Ar-Ar-R}? Also compare
  with the standard deviation and standard error of the mean age of
  Mount Dromedary apatite.

\end{enumerate}
