\chapter{Exercises}
\label{ch:exercises}

\begin{enumerate}

\item If helium ions (mass number = 4) are accelerated with a voltage
  of 10 kV in a mass spectrometer, at what speed are they emitted from
  the source? Recall that 1 atomic mass unit (amu) = 1.66 $\times$
  10$^{-27}$ kg and the elementary charge is 1.602 $\times$ 10$^{-19}$
  C. \rotatebox[origin=c]{180}{[Answer: 695 km/s]}

\item We are trying to estimate the Rb concentration in a rock for the
  purpose of whole rock Rb-Sr dating. To this end, we use a spike with
  a Rb concentration of 7.5 ppm containing 99.4\% $^{87}$Rb and 0.6\%
  $^{85}$Rb (these are atomic abundances). We mix 3.5g of the spike
  with 0.25g of sample dissolved in 50g. The $^{87}$Rb/$^{85}$Rb-ratio
  of the mixture is 1.55, as measured by mass spectrometry. What is
  the Rb concentration in ppm? Note that natural Rb comprises of
  27.825\% $^{87}$Rb and 72.165\% $^{85}$Rb.  
  \rotatebox[origin=c]{180}{[Answer: 120 ppm]}

\item A biotite contains 465ppm Rb and 30ppm Sr with a
  $^{87}$Sr/$^{86}$Sr-ratio of 2.50.  Given an initial
  $^{87}$Sr/$^{86}$Sr-ratio of 0.7035, what is the age of the biotite?
  Natural Rb has an atomic mass of 85.4678 and comprises 72.165\%
  $^{85}$Rb and 27.825\% $^{87}$Rb, which has a half life of t$_{1/2}$
  = 48.8 Gyr. Sr has an atomic mass of 87.62. Its non-radiogenic
  isotopes occur with the following abundances: $^{84}$Sr/$^{86}$Sr =
  0.056584 and $^{86}$Sr/$^{88}$Sr = 0.1194.
  \rotatebox[origin=c]{180}{[Answer: 2.36 Ga]}

\item Consider a zircon with the following composition: U = 792.1 ppm,
  Th = 318.6 ppm, Pb = 208.2 ppm. Atomic masses for U, Th and Pb in
  the zircon are 238.04, 232.04 and 205.94, respectively. The isotopic
  composition of the Pb is as follows: $^{204}$Pb = 0.048\%,
  $^{206}$Pb = 80.33\%, $^{207}$Pb = 9.00\%, $^{208}$Pb =
  10.63\%. Assume the following initial Pb composition: 204 : 206 :
  207 : 208 = 1.00 : 16.25 : 15.51 : 35.73. The decay constants for
  $^{238}$U, $^{235}$U and $^{232}$Th are given in Section
  \ref{sec:U-Pb}. $^{238}$U/$^{235}$U = 137.818. Calculate the
  $^{206}$Pb/$^{238}$U, $^{207}$Pb/$^{235}$U, $^{208}$Pb/$^{232}$Th
  and $^{207}$Pb/$^{206}$Pb-age of the zircon. Give an account of its
  formation history. \\ \rotatebox[origin=c]{180}{[Answer: 1.405,
      1.523, 1.284 and 1.689 Ga]}

\item A biotite was separated from granite and dated with the K-Ar
  method. The analytical data are as follows: K$_2$O = 8.45 weight \%,
  radiogenic $^{40}$Ar = 6.015 $\times$ 10$^{-10}$ mol/g.  What is the
  K-Ar age of the biotite? The atomic mass of K is 39.098 (and oxygen
  15.9994), with an isotopic composition that comprises 93.258\%
  $^{39}$K, 6.730\% $^{41}$K and 0.01167\% $^{40}$K, which has a
  half-life of t$_{1/2}$ = 1.248 Gyr. Recall that only 10.72\% of the
  $^{40}$K decays to $^{40}$Ar, with the remaining 89.28\% turning
  into $^{40}$Ca. \rotatebox[origin=c]{180}{[Answer: 47.7 Ma]}

\item Consider a biotite with a conventional K-Ar age of 384Ma. A
  $^{40}$Ar/$^{39}$Ar step-heating experiment yields the following
  data:

\begin{centering}
\begin{tabular}{c@{~}|@{~}c@{~}c@{~}c@{~}c@{~}c@{~}c@{~}c}
\% $^{39}$Ar released & 7 & 15 & 20 & 25 & 35 & 70 & 100\\
$^{40}$Ar$^*/^{39}$Ar & 2.27 & 4.97 & 6.68 & 9.58 & 10.25 & 10.10 & 10.26 \\
\end{tabular}
\end{centering}

The analysis was done using a co-irradiated 1.062 Ga biotite age
standard yielding a $^{40}$Ar$^*$/$^{39}$Ar-ratio of 27.64.  Construct
the $^{40}$Ar/$^{39}$Ar age spectrum and use this to comment on the
thermal history of the host rock. t$_{1/2}$($^{40}$Ar) = 1.248
Gyr.\\ \rotatebox[origin=c]{180}{[Answer: 115, 243, 319, 442, 470, 463
    and 470 Ma]}

\item How many cm$^3$ of helium does a rock weighing 1 kg and
  containing 2 ppm of uranium produce after 1 billion years? The molar
  volume of an ideal gas is 22.414 litres. Uranium has an atomic mass
  of 238.04 with $^{238}$U/$^{235}$U = 137.818. Decay constants of U
  are given in Section \ref{sec:U-Pb}.
\rotatebox[origin=c]{180}{[Answer: 0.27 cm$^3$]}

\item Repeated analysis of the Fish Canyon zircon age standard (t=27.8
  Ma) yields the following fission track data:
  
\begin{centering}
\begin{tabular}{ccc}
$\rho_s$ & $\rho_i$ & $\rho_d$ \\
($\times$10$^5$cm$^{-2}$) & ($\times$10$^6$cm$^{-2}$) & ($\times$10$^5$cm$^{-2}$)\\
\hline
36.56 & 6.282 & 2.829 \\
38.97 & 7.413 & 3.313 \\
56.53 & 7.878 & 2.457 \\
41.05 & 8.578 & 3.485 \\
45.87 & 6.985 & 2.482
\end{tabular}
\end{centering}

Compute the average $\zeta$-calibration factor and use this to
calculate the zircon fission track ages of the following rocks:

\begin{centering}
\begin{tabular}{r@{~}c@{~}c@{~}c}
~ & $\rho_s$ & $\rho_i$ & $\rho_d$ \\
~ & ($\times$10$^5$cm$^{-2}$) & ($\times$10$^6$cm$^{-2}$) & ($\times$10$^5$cm$^{-2}$)\\
\hline
Tardree rhyolite & 60.49 & 2.66 & 1.519 \\
Bishop tuff & 6.248 & 1.299 & 0.081 \\
\end{tabular}
\end{centering}

The half-life of $^{238}$U is t$_{1/2}$ = 4.47 Gyr.\\
\rotatebox[origin=c]{180}{[Answer: Tardree -- 57 Ma; Bishop -- 643 ka]}

\item A fossil mollusc has been found in a Quaternary beach formation
  and its activity ratio measured as A($^{230}$Th) / A($^{238}$U) =
  0.6782. Determine the age of the fossil assuming that $\gamma_\circ$
  = 1.15 and given that the half lives of $^{230}$Th and $^{234}$U are
  75,380 and 245,500 years,
  respectively. \rotatebox[origin=c]{180}{[Answer: 100kyr]}

\end{enumerate}
