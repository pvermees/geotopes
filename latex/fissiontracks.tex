\begin{refsection}

\chapter{Fission tracks}\label{ch:fissiontracks}

\texttt{IsoplotR} accepts three types of fission track data:

\begin{enumerate}
\item{`EDM':} \textzeta, err[\textzeta],
  \textrho\textsubscript{D}, err[\textrho\textsubscript{D}], 
  N\textsubscript{s}, N\textsubscript{i}
\item{`ICP (\textzeta)':} \textzeta, err[\textzeta], spot size
  (\textmu{m}), U\textsubscript{1}, err[U\textsubscript{1}],
  U\textsubscript{2}, err[U\textsubscript{2}], $\ldots$,
  U\textsubscript{n}, err[U\textsubscript{n}]
\item{`ICP (absolute)':} spot size (\textmu{m}), U\textsubscript{1},
  err[U\textsubscript{1}], U\textsubscript{2},
  err[U\textsubscript{2}], $\ldots$, U\textsubscript{n},
  err[U\textsubscript{n}]
\end{enumerate}

\noindent \textzeta, err[\textzeta], \textrho\textsubscript{D},
err[\textrho\textsubscript{D}], and spot size are scalars, and
N\textsubscript{s}, N\textsubscript{i} and U\textsubscript{i} and
err[U\textsubscript{i}] are vectors. The three formats represent two
different approaches to fission track dating:

\begin{enumerate}
\item The External Detector Method (EDM) implements the neutron
  irradiation method that was previously introduced in
  Section~\ref{sec:fission-tracks}.
\item The `ICP' method uses LA-ICP-MS (see
  Section~\ref{sec:mass-specs}) to determine the
  \textsuperscript{238}U-content of fission track samples.  The main
  reasons for this change are the increased throughput achieved by not
  having to irradiate samples and the ease of double-dating apatite
  and zircon with the U-Pb method.
\end{enumerate}

\section{The external detector method}
\label{sec:EDM}

Recall the fission track age equation from
Section~\ref{sec:fission-tracks}:
\begin{equation}
t =
\frac{1}{\lambda_{38}}\ln\left(1+\frac{g_i}{g_s}
\lambda_{38}\zeta\rho_d\frac{N_s}{N_i}\right)
\label{eq:tzeta2}
\end{equation}

The fission track method has been a test bed of statistical approaches
that have subsequently been adopted by other chronometers. A case in
point is the radial plot, which is uniquely suited to deal with the
large and highly variable (`heteroscedastic') counting uncertainties
of fission track data. Spontaneous fission is accurately described by
a Poisson distribution. For young and/or uranium-poor samples, there
is a finite probability that the spontaneous track count is zero for
some grains.  To accommodate such zero values, it is customary to use
an arcsin transformation for radial plots instead of the usual
logarithmic transformation \citep{galbraith1990a}:
\begin{equation}
z_j = \arcsin\sqrt{\frac{N_{sj} + 3/8}{N_{sj}+N_{ij}+3/4}}
\label{eq:zj}
\end{equation}

\noindent and
\begin{equation}
\sigma_j = \frac{1}{2\sqrt{N_{sj}+N_{ij}+1/2}}
\label{eq:sj}
\end{equation}

The arithmetic mean is not a reliable estimator of the true age, for a
similar reason why the arithmetic mean is not the best estimator of
the average U--Th--He composition and age. The Poisson distribution is
negatively skewed, and this biases the arithmetic mean. The solution
is similar as for the U--Th--He method, namely:

\begin{enumerate}
\item Take the average logarithm of the spontaneous and induced
  track densities.
\item Compute the fission track age the corresponds to this average
  ratio.
\end{enumerate}

This procedure again gives rise to a `central age'. Fission track data
are often overdispersed with respect to the Poisson counting
uncertainties, and this dispersion carries important thermal history
information. It is therefore customary to compute the average track
density ratio using a model-3 style random effects model, in which the
true $\rho_s/\rho_i$-ratio is assumed to follow a log-normal
distribution with location parameter $\mu$ and scale parameter
$\sigma$ \citep{galbraith1993}:
\begin{equation}
\ln\left[\frac{\rho_s}{\rho_i}\right] \sim \mathcal{N}(\mu,\sigma^2)
\label{eq:logrhosrhoi}
\end{equation}

This model gives rise to a two-parameter log-likelihood function:
\begin{equation}
  \mathcal{L}(\mu,\sigma^2) = \sum\limits_{j=1}^{n}
  \ln f_j(\mu,\sigma^2)
\label{eq:Lcentral}
\end{equation}

\noindent where the probability mass function $f_j(\mu,\sigma^2)$ is
defined as:
\begin{equation}
  f_j(\mu,\sigma^2) = {{N_{sj}+N_{ij}}\choose{N_{sj}}}
  \int\limits_{-\infty}^{\infty} \frac{e^{\beta N_{sj}} \left( 1 +
    e^\beta \right)^{-N_{sj}-N_{ij}}} {\sigma\sqrt{2\pi}
    e^{(\beta-\mu)^2/(2\sigma^2)}} d\beta
  \label{eq:fjms}
\end{equation}

\noindent in which the fission track count ratios are subject to two
sources of variation: the Poisson counting uncertainty and an
`(over)dispersion' factor $\sigma$.  Maximising Eq. \ref{eq:Lcentral}
results in two estimates $\hat{\mu}$ and $\hat{\sigma}$ and their
respective standard errors.  Substituting $\exp[\hat{\mu}]$ for
$N_s/N_i$ in Equation~\ref{eq:tzeta2} produces the desired central
age. $\hat{\sigma}$ estimates the overdispersion, and quantifies the
excess scatter of the single grain ages which cannot be explained by
the Poisson counting statistics alone. This dispersion can be just as
informative as the central age itself, as it encodes geologically
meaningful information about the compositional heterogeneity and
cooling history of the sample.

\section{LA-ICP-MS based fission track dating}
\label{sec:ICPFT}

The EDM continues to be the most widely used analytical protocol in
fission track dating. However, over the past decade, an increasing
number of laboratories have abandoned it and switched to LA-ICP-MS as
a means of determining the uranium concentration of datable minerals,
thus reducing sample turnover time and removing the need to handle
radioactive materials\citep{hasebe2004, chew2012, vermeesch2017}. The
statistical analysis of ICP-MS based FT data is less straightforward
and less well developed than that of the EDM. The latter is based on
simple ratios of Poisson variables, and forms the basis of a large
edifice of statistical methods which cannot be directly applied to
ICP-MS based data. The following paragraphs summarise
\citet{vermeesch2017}'s solution to these issues, as implemented in
\texttt{IsoplotR}.\\

Two analytical approached are being used in ICP-based fission track
geochronology, which each correspond to a different data format:

\begin{enumerate}
\item The `absolute dating' method is based on
  Equation~\ref{eq:tFT}:
  \begin{equation}
    \hat{t} = \frac{1}{\lambda_D}
    \ln \left(1 + \frac{\lambda_D}{\lambda_f}\frac{N_s}{[\hat{U}] A_s R_e ~ q}\right)
    \label{eq:tICP}
  \end{equation}

  where $N_s$ is the number of spontaneous tracks counted over an area
  $A_s$, $q$ is an `efficiency factor' \citep[$\sim$0.93 for apatite and
    $\sim$1 for
    zircon,][]{iwano1998,enkelmann2003,jonckheere2003b,soares2013} and
  $[\hat{U}]$ is the $^{238}$U-concentration (in atoms per unit volume)
  measured by LA-ICP-MS. Equation \ref{eq:tICP} requires an explicit
  value for $\lambda_f$ and assumes that the etchable range ($R_e$) is
  accurately known \citep{soares2014}.

\item The \textzeta-calibration method folds the etch efficience,
  decay constant and etchable range into a $\zeta$-calibration factor
  akin to that used in the EDM:
  \begin{equation}
    \hat{t} = \frac{1}{\lambda_D}
    \ln \left(1+\frac{1}{2}\lambda_D\hat{\zeta_{ICP}}\frac{N_s}{A_s[\hat{U}]}\right)
    \label{eq:tzetahatICP}
  \end{equation}

  in which $\hat{\zeta_{ICP}}$ is determined by analysing a standard of
  known FT age \citep{hasebe2004}. Note that, in contrast with the
  `absolute' dating method, the $\zeta$-calibration method allows
  $[\hat{U}]$ to be expressed in any concentration units (e.g., ppm or
  wt\% of total U) or could even be replaced with the measured U/Ca-,
  U/Si- or U/Zr-ratios produced by the ICP-MS instrument.
\end{enumerate}

\section{Compositional zoning}\label{sec:zoning}

Uranium-bearing minerals such as apatite and zircon often exhibit
compositional zoning, which must either be removed or quantified in
order to ensure unbiased ages. Two approaches are being used to deal
with this issue:

\begin{enumerate}
\item The effect of compositional zoning can be \emph{removed} by
  covering the entire counting area with one large laser spot
  \citep{soares2014} or a raster \citep{hasebe2004}. $s[\hat{U}]$ is
  then simply given by the analytical uncertainty of the LA-ICP-MS
  instrument, which typically is an order of magnitude lower than
  the standard errors of induced track counts in the EDM.
\item Alternatively, the uranium-heterogeneity can be
  \emph{quantified} by analysing multiple spots per analysed grain
  \citep{hasebe2009}. This is why \texttt{IsoplotR} accommodates
  multiple uranium measurements per aliquot (U\textsubscript{1},
  err[U\textsubscript{1}], $\ldots$, U\textsubscript{n},
  err[U\textsubscript{n}])\\

  It is commonly found that the variance of the different
  uranium-measurements within each grain far exceeds the formal
  analytical uncertainty of each spot measurement. \texttt{IsoplotR}
  assumes that this dispersion is constant across all aliquots and
  follows a log-normal distribution:
  \begin{equation}
    \ln\![U_{jk}] \sim \mathcal{N}(\mu_j,\sigma^2+s[U]_{jk}^2)
    \label{eq:lognorm}
  \end{equation}

  where $\hat{U}_{jk}$ is the $k$\textsuperscript{th} (out of $n_j$)
  uranium concentration measurement, $s[U]_{jk}$ is its standard
  error, and and $\mu_j$ and $\sigma^2$ are the (unknown) mean and
  variance of a Normal distribution. Note that $\mu_j$ is allowed to
  vary from grain to grain but $\sigma$ is not.  $\mu_j$ and $\sigma$
  can be estimated using the method of maximum likelihood, and the
  corresponding geometric mean uranium concentrations directly plugged
  into Equation~\ref{eq:tICP}.
\end{enumerate}

\section{Zero track counts}
\label{sec:zeroICP}

In contrast with the EDM, ICP-MS based FT data do not offer an easy
way to deal with zero track counts. One pragmatic solution to this
problem is to approximate the ICP-MS based uranium concentration
measurement with an `equivalent induced track density', using the
following linear transformation:
\begin{equation}
\hat{N}_{ij} = \rho_j A_{sj} [\hat{U}_j]
\end{equation}

where $A_{sj}$ is the area over which the spontaneous tracks of the
j$^{th}$ grain have been counted and $\rho_j$ plays a similar role as
$\rho_d$ in Eq.~\ref{eq:tzeta}. From the requirement that the variance
of a Poisson-distributed variable equals its mean, it follows that:
\begin{equation}
\hat{N}_{ij} = \rho_j^2 A_{sj}^2 s[\hat{U}_j]^2
\end{equation}

\noindent from which it is easy to determine $\rho_j$. Thus the ICP-MS
data have effectively been converted in to EDM data, and can be
subjected to the same treatment as EDM data.

\section{TODO}

To plot ICP-MS based fission track data on a radial plot, we can
replace Eqs.~\ref{eq:zj} and \ref{eq:sj} with

\begin{align}
  z_j & = \ln (\hat{t}_j) \mbox{,}   \label{eq:zj2} \\
  \mbox{and~} s_j & = \sqrt{ 
    \left(\frac{s[\hat{\zeta}]}{\hat{\zeta}}\right)^2 +
    \left(\frac{s[\hat{U}]}{\hat{U}}\right)^2 +
    \frac{1}{N_s}
  }   \label{eq:sj2}
\end{align}

respectively \citep{galbraith2010b}. Alternatively, a square root
transformation may be more appropriate for young and/or U-poor samples
(Galbraith, \textit{pers. commun.}):
\begin{align}
  z_j & = \sqrt{\hat{t}_j} \mbox{,}   \label{eq:zj3} \\
  \mbox{and~} s_j & = s[\hat{t}_j]\bigg/\left(2\sqrt{\hat{t}_j}\right)
  \label{eq:sj3}
\end{align}


Equation~\ref{eq:tzeta2} represents a classic case of a `matched
pairs' experimental design \citep{galbraith2010}. By counting the
spontaneous and induced tracks over exactly the same area, the age
calculation reduces to a simple comparison of two Poisson-distributed
variables ($N_s$ and $N_i$). This enables an explicit maximum
likelihood formulation, which greatly simplifies all subsequent
statistical analyses. As a result, it is fair to say that the fission
track method represents the gold standard among geochronometers in
terms of statistical rigour.\\

Abandoning the EDM in favour of LA-ICP-MS sacrifices this
methodological elegance and causes problems in dealing with uranium
zoning and zero-track grains. \texttt{IsoplotR} solves these problems
using methods proposed by \citet{vermeesch20017}:

\begin{enumerate}
\item{Uranium zoning}: suppose that the analyst has placed a round
  laser spot in the top half of the strongly zoned grain. This would
  result in a high uranium concentration (or isotopic ratio
  measurement) and a small analytical uncertainty, but would be
  completely unrepresentative of the average composition of the
  grain. Blindly combining such a single spot measurement with the
  number of spontaneous fission tracks counted over the entire crystal
  would produce a precise but grossly inaccurate age.

\end{enumerate}

A first option is to only count the spontaneous fission tracks that
are located within the laser ablation spot, and to plug the resulting
track counts, areas and \textsuperscript{238}U-concentrations into
Equation \ref{eq:FT}.  Matching the areas over which
N\textsubscript{s} and \textsuperscript{238}U are measured reduces the
detrimental effect of (lateral) U-zoning on the fission track age
accuracy. The main limitation of matching the areas is a reduction in
precision due to the low number of spontaneous tracks counted within
the outline of a small ablation pit. This problem can be circumvented
by acquiring multiple \textsuperscript{238}U-measurements per grain.\\

In a second approach to LAICPMS-based fission track dating,
\texttt{IsoplotR} then jointly considers all the grains that have been
analysed multiple times to quantify the degree of U-zoning within the
grains.\\

The accuracy of the two `absolute dating' approaches discussed thus
far is fundamentally limited by the accuracy of the U-concentration
measurements, the fission track decay constant and the etching and
counting efficiencies.  Unfortunately, all these factors are
potentially affected by unquantifiable biases.\\

The third (using a single laser spot per grain) and fourth (using
multiple spots) approach to LAICPMS-based fission track dating remove
these systematic errors by normalising to a standard of known fission
track age and defining a new `zeta' calibration constant
$\zeta_{icp}$:

\begin{equation}
  t = \frac{1}{\lambda_{238}} \ln\left( 1 +
  \frac{\lambda_{238}\zeta_{icp}}{2} \frac{N_s}{[{}^{238}U] A_s} \right)
  \label{eq:tICPzeta}
\end{equation}

\noindent where $[{}^{238}U]$ may either stand for the
\textsuperscript{238}U-concentration (in ppm) \emph{or} for the U/Ca
(for apatite) or U/Si (for zircon) ratio measurement, and $A_s$ is the
spontaneous track counting area.  \texttt{IsoplotR} implements the
four approaches to LAICPMS-based fission track dating by giving the
user the choice between an `absolute' and `$\zeta$-calibration'
option, and between one or more U-measurements per grain.\\

The zero track problem is solved by converting the U-concentration
measurements into a `virtual' induced track density ($\hat{N}_i$) by
replacing $[{}^{238}U] A_s$ with $\hat{N}_i/\rho_{icp}$ in
Equation~\ref{eq:tICPzeta}, so that the Poisson uncertainty of
$\hat{N}_i$ matches the uncertainty of the U-concentration measurement
\citep{vermeesch2017}.

\printbibliography[heading=subbibliography]

\end{refsection}
