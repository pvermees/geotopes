\chapter{Introduction}
\label{ch:intro2}

Part~1 of these notes gave a very basic introduction to geochronology.
At this basic level, it is possible to write one's own data processing
software from scratch, as we have done in the \texttt{R} practicals.
Unfortunately this is not so easy at a more advanced level.
Research-grade geochronological data processing chains involve several
layers of highly specialised software packages:

\begin{enumerate}
\item A first layer of software controls the mass spectrometer and
  extracts the raw time resolved isotopic signals from it. This
  software generally comes with the mass spectrometer, and was written
  and designed by engineers who may be completely unfamiliar with the
  geological applications of the equipment.
\item The output files from this low level software are passed on to a
  second layer of software, which processes the raw mass spectrometer,
  combines standard with standards, performs isotope dilution
  calculations, etc. This `middleware' is sometimes written by
  geologists, and sometimes by companies. Examples are
  \texttt{Iolite}, \texttt{GLITTER}, \texttt{Squid}, \texttt{LADR} and
  \texttt{ET\_Redux} for U--Pb geochronology; \texttt{ArArCalc} and
  \texttt{Pychron} for Ar--Ar geochronology, etc.
\item The output files produced by the second layer of data processing
  software require further processing for more advanced statistical
  analysis and visualisation. \texttt{IsoplotR} is a software package
  that fulfils this role.
\end{enumerate}

Chapter~\ref{ch:intro2IsoplotR} provides a brief introduction to the
design philosophy and operating principles of \texttt{IsoplotR}, which
will be explored further in later chapters.  Chapter~\ref{ch:formats}
introduces the important subject of error correlations, and shows how
these are captured by \texttt{IsoplotR}'s different input formats.  We
will see that error correlations plays a fundamental role in all of
\texttt{IsoplotR}'s methods.  Chapter~\ref{ch:regression} reviews the
subject of linear regression, which underpins the construction of
isochrons. \texttt{IsoplotR} currently implements three different
types of error weighted linear regression algorithms that account for
error correlations between variables and between aliquots in two or
three dimensions.  Chapter~\ref{ch:overdispersion} explains how these
three methods represent different approaches of dealing with
overdispersion.  Chapter~\ref{ch:averaging} introduces a weighted mean
plot to visualise multiple age estimates and proposes a heuristic
method to detect outliers.  Chapter~\ref{ch:confidence} presents three
approaches to construct confidence intervals for isochron ages,
weighted means and so forth.  It introduces a profile log-likelihood
method for the calculation of asymmetric confidence intervals.\\

Chapter~\ref{ch:density} discusses three further methods to visualise
multi-aliquot collections of ages. Cumulative age distributions (CADs)
and kernel density estimates (KDEs) show the frequency distribution of
the age measurements but do not explicitly take into account the
analytical uncertainties. The radial plot is introduced as a more
appropriate data visualisation tool for `heteroscedastic' data
(i.e. data with unequal measurements uncertainties). The radial plot
provides a good vehicle to assess the dispersion of multi-aliquot
datasets. Overdispersed datasets require further processing with
continuous or discrete mixture models that are discussed in
Chapter~\ref{ch:mixtures}.\\

With these basic statistical building blocks in place, the remainder
of the notes cover issues that are specific to individual
geochronometers and their geological applications. They will be
discussed in the same order as they are listed in \texttt{IsoplotR}'s
graphical user interface.

Chapter~\ref{ch:UPb} provides an in-depth discussion of
\texttt{IsoplotR}'s U--Pb functionality. This includes an overview of
the various input formats, concordia ages, discordia regression,
common-Pb correction methods and initial disequilibrium corrections.
Chapter~\ref{ch:detritals} covers the subject of detrital U--Pb
geochronology, which includes a discussion of discordance filters,
maximum depositional age estimation and multidimensional scaling
analysis. Chapter~\ref{ch:USeries} covers U--Th dating and
Chapter~\ref{ch:thermochronology} thermochronology, including both the
traditional external detector method and the new LA-ICP-MS based
approach.\\

Finally, Section~\ref{sec:future} sets out a roadmap for future
developments to improve the accuracy and precision of geochronological
data, and to provide closer integration of \texttt{IsoplotR} with
earlier steps of the data processing chain.
