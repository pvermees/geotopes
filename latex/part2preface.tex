\begin{refsection}
\chapter{Introduction}
\label{ch:intro2}

Part~1 of these notes gave a very basic introduction to geochronology.
At this basic level, it is possible to write one's own data processing
software from scratch, as we have done in the \texttt{R} practicals.
Unfortunately this is not so easy at a more advanced level.
Research-grade geochronological data processing chains involve several
layers of highly specialised software packages:

\begin{enumerate}
\item A first layer of software controls the mass spectrometer and
  extracts the raw time resolved isotopic signals from it. This
  software generally comes with the mass spectrometer, and was written
  and designed by engineers who may be completely unfamiliar with the
  geological applications of the equipment.
\item The output files from this low level software are passed on to a
  second layer of software, which processes the raw mass spectrometer,
  combines standard with standards, performs isotope dilution
  calculations, etc. This `middleware' is sometimes written by
  geologists, and sometimes by companies. Examples are
  \texttt{Iolite}, \texttt{GLITTER}, \texttt{Squid}, \texttt{LADR} and
  \texttt{ET\_Redux} for U--Pb geochronology; \texttt{ArArCalc} and
  \texttt{Pychron} for Ar--Ar geochronology, etc.
\item The output files produced by the second layer of data processing
  software require further processing for more advanced statistical
  analysis and visualisation. \texttt{IsoplotR} is a software package
  that fulfils this role.
\end{enumerate}

Part~2 of these notes provide an overview of the mathematical
underpinnings of \texttt{IsoplotR}. The structure of this collection
of chapters is very similar to that of Part~3, which contains a
detailed manual of \texttt{IsoplotR}'s graphical and command line user
interfaces. Chapters~\ref{ch:statistics} and \ref{ch:generic} discuss
some basic statistical principles and generic plot devices on which
all the subsequent analytical devices are based. This is followed by a
systematic discussion of all the chronometers in the toolbox,
including U--Pb (Chapter~\ref{ch:U-Pb}); Pb--Pb and Th--Pb
(Chapter~\ref{ch:Th-Pb-Pb}); Ar--Ar and K--Ca
(Chapter~\ref{ch:ArArKCa}); Rb--Sr, Sm--Nd, Lu--Hf and Re--Os
(Chapter~\ref{ch:PD}); U--Th--(Sm)--He (Chapter~\ref{ch:UThHe});
fission tracks (Chapter~\ref{ch:fissiontracks}); U-series
disequilibrium dating (Chapter~\ref{ch:ThU}) and detrital
geochronology (Chapter~\ref{ch:detrital}).

%\printbibliography[heading=subbibliography]

\end{refsection}
