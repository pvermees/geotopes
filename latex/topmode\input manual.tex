isochron               package:IsoplotR                R Documentation

_C_a_l_c_u_l_a_t_e _a_n_d _p_l_o_t _i_s_o_c_h_r_o_n_s

_D_e_s_c_r_i_p_t_i_o_n:

     Plots cogenetic U-Pb, Ar-Ar, K-Ca, Pb-Pb, Th-Pb, Rb-Sr, Sm-Nd,
     Re-Os, Lu-Hf, U-Th-He or Th-U data as X-Y scatterplots, fits an
     isochron curve through them using the ‘york’, ‘titterington’ or
     ‘ludwig’ function, and computes the corresponding isochron age,
     including decay constant uncertainties.

_U_s_a_g_e:

     isochron(x, ...)
     
     ## Default S3 method:
     isochron(
       x,
       oerr = 3,
       sigdig = 2,
       show.numbers = FALSE,
       levels = NA,
       clabel = "",
       xlab = "x",
       ylab = "y",
       ellipse.fill = c("#00FF0080", "#FF000080"),
       ellipse.stroke = "black",
       ci.col = "gray80",
       line.col = "black",
       lwd = 1,
       plot = TRUE,
       title = TRUE,
       model = 1,
       show.ellipses = 1 * (model != 2),
       hide = NULL,
       omit = NULL,
       omit.fill = NA,
       omit.stroke = "grey",
       ...
     )
     
     ## S3 method for class 'UPb'
     isochron(
       x,
       oerr = 3,
       sigdig = 2,
       show.numbers = FALSE,
       levels = NA,
       clabel = "",
       joint = TRUE,
       ellipse.fill = c("#00FF0080", "#FF000080"),
       ellipse.stroke = "black",
       type = 1,
       ci.col = "gray80",
       line.col = "black",
       lwd = 1,
       plot = TRUE,
       exterr = FALSE,
       model = 1,
       show.ellipses = 1 * (model != 2),
       anchor = 0,
       hide = NULL,
       omit = NULL,
       omit.fill = NA,
       omit.stroke = "grey",
       ...
     )
     
     ## S3 method for class 'PbPb'
     isochron(
       x,
       oerr = 3,
       sigdig = 2,
       show.numbers = FALSE,
       levels = NA,
       clabel = "",
       ellipse.fill = c("#00FF0080", "#FF000080"),
       ellipse.stroke = "black",
       inverse = TRUE,
       ci.col = "gray80",
       line.col = "black",
       lwd = 1,
       plot = TRUE,
       exterr = TRUE,
       model = 1,
       growth = FALSE,
       show.ellipses = 1 * (model != 2),
       hide = NULL,
       omit = NULL,
       omit.fill = NA,
       omit.stroke = "grey",
       ...
     )
     
     ## S3 method for class 'ArAr'
     isochron(
       x,
       oerr = 3,
       sigdig = 2,
       show.numbers = FALSE,
       levels = NA,
       clabel = "",
       ellipse.fill = c("#00FF0080", "#FF000080"),
       ellipse.stroke = "black",
       inverse = TRUE,
       ci.col = "gray80",
       line.col = "black",
       lwd = 1,
       plot = TRUE,
       exterr = TRUE,
       model = 1,
       show.ellipses = 1 * (model != 2),
       hide = NULL,
       omit = NULL,
       omit.fill = NA,
       omit.stroke = "grey",
       ...
     )
     
     ## S3 method for class 'ThPb'
     isochron(
       x,
       oerr = 3,
       sigdig = 2,
       show.numbers = FALSE,
       levels = NA,
       clabel = "",
       ellipse.fill = c("#00FF0080", "#FF000080"),
       ellipse.stroke = "black",
       inverse = FALSE,
       ci.col = "gray80",
       line.col = "black",
       lwd = 1,
       plot = TRUE,
       exterr = TRUE,
       model = 1,
       show.ellipses = 1 * (model != 2),
       hide = NULL,
       omit = NULL,
       omit.fill = NA,
       omit.stroke = "grey",
       ...
     )
     
     ## S3 method for class 'KCa'
     isochron(
       x,
       oerr = 3,
       sigdig = 2,
       show.numbers = FALSE,
       levels = NA,
       clabel = "",
       inverse = FALSE,
       ci.col = "gray80",
       ellipse.fill = c("#00FF0080", "#FF000080"),
       ellipse.stroke = "black",
       line.col = "black",
       lwd = 1,
       plot = TRUE,
       exterr = TRUE,
       model = 1,
       show.ellipses = 1 * (model != 2),
       hide = NULL,
       omit = NULL,
       omit.fill = NA,
       omit.stroke = "grey",
       ...
     )
     
     ## S3 method for class 'RbSr'
     isochron(
       x,
       oerr = 3,
       sigdig = 2,
       show.numbers = FALSE,
       levels = NA,
       clabel = "",
       ellipse.fill = c("#00FF0080", "#FF000080"),
       ellipse.stroke = "black",
       inverse = FALSE,
       ci.col = "gray80",
       line.col = "black",
       lwd = 1,
       plot = TRUE,
       exterr = TRUE,
       model = 1,
       show.ellipses = 1 * (model != 2),
       hide = NULL,
       omit = NULL,
       omit.fill = NA,
       omit.stroke = "grey",
       ...
     )
     
     ## S3 method for class 'ReOs'
     isochron(
       x,
       oerr = 3,
       sigdig = 2,
       show.numbers = FALSE,
       levels = NA,
       clabel = "",
       ellipse.fill = c("#00FF0080", "#FF000080"),
       ellipse.stroke = "black",
       inverse = FALSE,
       ci.col = "gray80",
       line.col = "black",
       lwd = 1,
       plot = TRUE,
       exterr = TRUE,
       model = 1,
       show.ellipses = 1 * (model != 2),
       hide = NULL,
       omit = NULL,
       omit.fill = NA,
       omit.stroke = "grey",
       ...
     )
     
     ## S3 method for class 'SmNd'
     isochron(
       x,
       oerr = 3,
       sigdig = 2,
       show.numbers = FALSE,
       levels = NA,
       clabel = "",
       ellipse.fill = c("#00FF0080", "#FF000080"),
       ellipse.stroke = "black",
       inverse = FALSE,
       ci.col = "gray80",
       line.col = "black",
       lwd = 1,
       plot = TRUE,
       exterr = TRUE,
       model = 1,
       show.ellipses = 1 * (model != 2),
       hide = NULL,
       omit = NULL,
       omit.fill = NA,
       omit.stroke = "grey",
       ...
     )
     
     ## S3 method for class 'LuHf'
     isochron(
       x,
       oerr = 3,
       sigdig = 2,
       show.numbers = FALSE,
       levels = NA,
       clabel = "",
       ellipse.fill = c("#00FF0080", "#FF000080"),
       ellipse.stroke = "black",
       inverse = FALSE,
       ci.col = "gray80",
       line.col = "black",
       lwd = 1,
       plot = TRUE,
       exterr = TRUE,
       model = 1,
       show.ellipses = 1 * (model != 2),
       hide = NULL,
       omit = NULL,
       omit.fill = NA,
       omit.stroke = "grey",
       ...
     )
     
     ## S3 method for class 'UThHe'
     isochron(
       x,
       sigdig = 2,
       oerr = 3,
       show.numbers = FALSE,
       levels = NA,
       clabel = "",
       ellipse.fill = c("#00FF0080", "#FF000080"),
       ellipse.stroke = "black",
       ci.col = "gray80",
       line.col = "black",
       lwd = 1,
       plot = TRUE,
       model = 1,
       show.ellipses = 2 * (model != 2),
       hide = NULL,
       omit = NULL,
       omit.fill = NA,
       omit.stroke = "grey",
       ...
     )
     
     ## S3 method for class 'ThU'
     isochron(
       x,
       type = 2,
       oerr = 3,
       sigdig = 2,
       show.numbers = FALSE,
       levels = NA,
       clabel = "",
       ellipse.fill = c("#00FF0080", "#FF000080"),
       ellipse.stroke = "black",
       ci.col = "gray80",
       line.col = "black",
       lwd = 1,
       plot = TRUE,
       exterr = TRUE,
       model = 1,
       show.ellipses = 1 * (model != 2),
       hide = NULL,
       omit = NULL,
       omit.fill = NA,
       omit.stroke = "grey",
       y0option = 4,
       ...
     )
     
_A_r_g_u_m_e_n_t_s:

       x: EITHER a matrix with the following five columns:

          ‘X’: the x-variable

          ‘sX’: the standard error of ‘X’

          ‘Y’: the y-variable

          ‘sY’: the standard error of ‘Y’

          ‘rXY’: the correlation coefficient of ‘X’ and ‘Y’

          OR

          an object of class ‘ArAr’, ‘KCa’, ‘PbPb’, ‘UPb’, ‘ThPb’,
          ‘ReOs’, ‘RbSr’, ‘SmNd’, ‘LuHf’, ‘UThHe’ or ‘ThU’.

     ...: optional arguments to be passed on to the generic plot
          function if ‘model=2’

    oerr: indicates whether the analytical uncertainties of the output
          are reported in the plot title as:

          ‘1’: 1sigma absolute uncertainties.

          ‘2’: 2sigma absolute uncertainties.

          ‘3’: absolute (1-alpha)% confidence intervals, where alpha
          equales the value that is stored in ‘settings('alpha')’.

          ‘4’: 1sigma relative uncertainties (\%).

          ‘5’: 2sigma relative uncertainties (\%).

          ‘6’: relative (1-alpha)% confidence intervals, where alpha
          equales the value that is stored in ‘settings('alpha')’.

  sigdig: the number of significant digits of the numerical values
          reported in the title of the graphical output

show.numbers: logical flag (‘TRUE’ to show grain numbers)

  levels: a vector with additional values to be displayed as different
          background colours within the error ellipses.

  clabel: label for the colour scale

    xlab: text label for the horizontal plot axis

    ylab: text label for the vertical plot axis

ellipse.fill: Fill colour for the error ellipses. This can either be a
          single colour or multiple colours to form a colour ramp.
          Examples:

          a single colour: ‘rgb(0,1,0,0.5)’, ‘'#FF000080'’, ‘'white'’,
          etc.;

          multiple colours: ‘c(rbg(1,0,0,0.5)’, ‘rgb(0,1,0,0.5))’,
          ‘c('#FF000080','#00FF0080')’, ‘c('blue','red')’,
          ‘c('blue','yellow','red')’, etc.;

          a colour palette: ‘rainbow(n=100)’,
          ‘topo.colors(n=100,alpha=0.5)’, etc.; or

          a reversed palette: ‘rev(topo.colors(n=100,alpha=0.5))’, etc.

          For empty ellipses, set ‘ellipse.col=NA’

ellipse.stroke: the stroke colour for the error ellipses. Follows the
          same formatting guidelines as ‘ellipse.fill’

  ci.col: the fill colour for the confidence interval of the intercept
          and slope.

line.col: colour of the isochron line

     lwd: line width

    plot: if ‘FALSE’, suppresses the graphical output

   title: add a title to the plot?

   model: construct the isochron using either:

          ‘1’: Error-weighted least squares regression

          ‘2’: Ordinary least squares regression

          ‘3’: Error-weighted least squares with overdispersion term

show.ellipses: show the data as:

          ‘0’: points

          ‘1’: error ellipses

          ‘2’: error crosses

    hide: vector with indices of aliquots that should be removed from
          the plot.

    omit: vector with indices of aliquots that should be plotted but
          omitted from the isochron age calculation.

omit.fill: fill colour that should be used for the omitted aliquots.

omit.stroke: stroke colour that should be used for the omitted
          aliquots.

   joint: logical. Only applies to U-Pb data formats 4 and above. If
          ‘TRUE’, carries out three dimensional regression.  If
          ‘FALSE’, uses two dimensional isochron regression.  The
          latter can be used to compute {}^{207}Pb/{}^{235}U isochrons,
          which are immune to the complexities of initial
          {}^{234}U/{}^{238}U disequilibrium.

    type: if ‘x’ has class ‘UPb’ and ‘x$format=4’, ‘5’ or ‘6’:

          ‘1’: ^{204}Pb/^{206}Pb vs. ^{238}U/^{206}Pb

          ‘2’: ^{204}Pb/^{207}Pb vs. ^{235}U/^{207}Pb

          if ‘x’ has class ‘UPb’ and ‘x$format=7’ or ‘8’:

          ‘1’: ^{208}Pb{}_\circ/^{206}Pb vs. ^{238}U/^{206}Pb

          ‘2’: ^{208}Pb{}_\circ/^{207}Pb vs. ^{235}U/^{207}Pb

          ‘3’: ^{206}Pb{}_\circ/^{208}Pb vs. ^{232}Th/^{208}Pb

          ‘4’: ^{207}Pb{}_\circ/^{208}Pb vs. ^{232}Th/^{208}Pb

          if ‘x’ has class ‘ThU’, and following the classification of
          Ludwig and Titterington (1994), one of either:

          ‘1’: `Rosholt type-II' isochron, setting out
          ^{230}Th/^{232}Th vs. ^{238}U/^{232}Th

          ‘2’: `Osmond type-II' isochron, setting out ^{230}Th/^{238}U
          vs. ^{232}Th/^{238}U

          ‘3’: `Rosholt type-II' isochron, setting out ^{234}U/^{232}Th
          vs. ^{238}U/^{232}Th

          ‘4’: `Osmond type-II' isochron, setting out ^{234}U/^{238}U
          vs. ^{232}Th/^{238}U

  exterr: propagate external sources of uncertainty (J, decay
          constant)?

  anchor: control parameters to fix the intercept age or common Pb
          composition of the isochron fit. This can be a scalar or a
          vector.

          If ‘anchor[1]=0’: do not anchor the isochron.

          If ‘anchor[1]=1’: fix the common Pb composition at the values
          stored in ‘settings('iratio',...)’.

          If ‘anchor[1]=2’: force the isochron line to intersect the
          concordia line at an age equal to ‘anchor[2]’.

 inverse: toggles between normal and inverse isochrons. If the isochron
          plots ‘Y’ against ‘X’, and

          If ‘inverse=TRUE’, then ‘X’ = {}^{204}Pb/{}^{206}Pb and ‘Y’ =
          {}^{207}Pb/{}^{206}Pb (if ‘x’ has class ‘PbPb’), or ‘X’ =
          {}^{232}Th/{}^{208}Pb and ‘Y’ = {}^{204}Pb/{}^{208}Pb (if ‘x’
          has class ‘ThPb’), or ‘X’ = {}^{39}Ar/{}^{40}Ar and ‘Y’ =
          {}^{36}Ar/{}^{40}Ar (if ‘x’ has class ‘ArAr’), or ‘X’ =
          {}^{40}K/{}^{40}Ca and ‘Y’ = {}^{44}Ca/{}^{40}Ca (if ‘x’ has
          class ‘KCa’), or ‘X’ = {}^{87}Rb/{}^{87}Sr and ‘Y’ =
          {}^{86}Sr/{}^{87}Sr (if ‘x’ has class ‘RbSr’), or ‘X’ =
          {}^{147}Sm/{}^{143}Nd and ‘Y’ = {}^{144}Nd/{}^{143}Nd (if ‘x’
          has class ‘SmNd’), or ‘X’ = {}^{187}Re/{}^{187}Os and ‘Y’ =
          {}^{188}Os/{}^{187}Os (if ‘x’ has class ‘ReOs’), or ‘X’ =
          {}^{176}Lu/{}^{176}Hf and ‘Y’ = {}^{177}Hf/{}^{176}Hf (if ‘x’
          has class ‘LuHf’).

          If ‘inverse=FALSE’, then ‘X’ = {}^{206}Pb/{}^{204}Pb and ‘Y’
          = {}^{207}Pb/{}^{204}Pb (if ‘x’ has class ‘PbPb’), or ‘X’ =
          {}^{232}Th/{}^{204}Pb and ‘Y’ = {}^{208}Pb/{}^{204}Pb (if ‘x’
          has class ‘ThPb’), or ‘X’ = {}^{39}Ar/{}^{36}Ar and ‘Y’ =
          {}^{40}Ar/{}^{36}Ar (if ‘x’ has class ‘ArAr’), or ‘X’ =
          {}^{40}K/{}^{44}Ca and ‘Y’ = {}^{40}Ca/{}^{44}Ca (if ‘x’ has
          class ‘KCa’), or ‘X’ = {}^{87}Rb/{}^{86}Sr and ‘Y’ =
          {}^{87}Sr/{}^{86}Sr (if ‘x’ has class ‘RbSr’), or ‘X’ =
          {}^{147}Sm/{}^{144}Nd and ‘Y’ = {}^{143}Nd/{}^{144}Nd (if ‘x’
          has class ‘SmNd’), or ‘X’ = {}^{187}Re/{}^{188}Os and ‘Y’ =
          {}^{187}Os/{}^{188}Os (if ‘x’ has class ‘ReOs’), or ‘X’ =
          {}^{176}Lu/{}^{177}Hf and ‘Y’ = {}^{176}Hf/{}^{177}Hf (if ‘x’
          has class ‘LuHf’).

  growth: add Stacey-Kramers Pb-evolution curve to the plot?

y0option: controls the type of activity ratio that is reported along
          with the 3D isochron age. Only relevant to Th-U data formats
          1 and 2. Set to:

          ‘y0option=1’ for the authigenic ^{234}U/^{238}U activity
          ratio,

          ‘y0option=2’ for the detrital ^{230}Th/^{232}Th activity
          ratio,

          ‘y0option=3’ for the authigenic ^{230}Th/^{238}U activity
          ratio,

          ‘y0option=4’ for the initial ^{234}U/^{238}U activity ratio.

_D_e_t_a_i_l_s:

     Given several aliquots from a single sample, isochrons allow the
     non-radiogenic component of the daughter nuclide to be quantified
     and separated from the radiogenic component. In its simplest form,
     an isochron is obtained by setting out the amount of radiogenic
     daughter against the amount of radioactive parent, both normalised
     to a non-radiogenic isotope of the daughter element, and fitting a
     straight line through these points by least squares regression
     (Nicolaysen, 1961). The slope and intercept then yield the
     radiogenic daughter-parent ratio and the non-radiogenic daughter
     composition, respectively. There are several ways to fit an
     isochron.  The easiest of these is ordinary least squares
     regression, which weighs all data points equally. In the presence
     of quantifiable analytical uncertainty, it is equally
     straightforward to use the inverse of the y-errors as weights.  It
     is significantly more difficult to take into account uncertainties
     in both the x- and the y-variable (York, 1966). ‘IsoplotR’ does so
     for its U-Th-He isochron calculations. The York (1966) method
     assumes that the analytical uncertainties of the x- and
     y-variables are independent from each other. This assumption is
     rarely met in geochronology.  York (1968) addresses this issue
     with a bivariate error weighted linear least squares algorithm
     that accounts for covariant errors in both variables. This
     algorithm was further improved by York et al. (2004) to ensure
     consistency with the maximum likelihood approach of Titterington
     and Halliday (1979).

     ‘IsoplotR’ uses the York et al. (2004) algorithm for its Ar-Ar,
     K-Ca, Pb-Pb, Th-Pb, Rb-Sr, Sm-Nd, Re-Os and Lu-Hf isochrons. The
     maximum likelihood algorithm of Titterington and Halliday (1979)
     was generalised from two to three dimensions by Ludwig and
     Titterington (1994) for U-series disequilibrium dating. Also this
     algorithm is implemented in ‘IsoplotR’. Finally, the constrained
     maximum likelihood algorithms of Ludwig (1998) and Vermeesch
     (2020) are used for isochron regression of U-Pb data. The extent
     to which the observed scatter in the data can be explained by the
     analytical uncertainties can be assessed using the Mean Square of
     the Weighted Deviates (MSWD, McIntyre et al., 1966), which is
     defined as:

     MSWD = ([X - \hat{X}] Sigma_{X}^{-1} [X - \hat{X}]^T)/df

     where X are the data, \hat{X} are the fitted values, and Sigma_X
     is the covariance matrix of X, and df = k(n-1) are the degrees of
     freedom, where k is the dimensionality of the linear fit. MSWD
     values that are far smaller or greater than 1 indicate under- or
     overdispersed measurements, respectively. Underdispersion can be
     attributed to overestimated analytical uncertainties. ‘IsoplotR’
     provides three alternative strategies to deal with overdispersed
     data:

       1. Attribute the overdispersion to an underestimation of the
          analytical uncertainties. In this case, the excess scatter
          can be accounted for by inflating those uncertainties by a
          _factor_ sqrt{MSWD}.

       2. Ignore the analytical uncertainties and perform an ordinary
          least squares regression.

       3. Attribute the overdispersion to the presence of `geological
          scatter'.  In this case, the excess scatter can be accounted
          for by adding an overdispersion _term_ that lowers the MSWD
          to unity.

_V_a_l_u_e:

     If ‘x’ has class ‘PbPb’, ‘ThPb’, ‘ArAr’, ‘KCa’, ‘RbSr’, ‘SmNd’,
     ‘ReOs’ or ‘LuHf’, or ‘UThHe’, returns a list with the following
     items:

     a the intercept of the straight line fit and its standard error.

     b the slope of the fit and its standard error.

     cov.ab the covariance of the slope and intercept

     df the degrees of freedom of the linear fit (df=n-2)

     y0 a two- or three-element list containing:

          ‘y’: the atmospheric ^{40}Ar/^{36}Ar or initial
          ^{40}Ca/^{44}Ca, ^{187}Os/^{188}Os, ^{87}Sr/^{87}Rb,
          ^{143}Nd/^{144}Nd, ^{176}Hf/^{177}Hf or ^{208}Pb/^{204}Pb
          ratio.

          ‘s[y]’: the standard error of ‘y’

          ‘disp[y]’: the standard error of ‘y’ enhanced by sqrt{mswd}
          (only applicable if ‘ model=1’).

     age a three-element list containing:

          ‘t’: the ^{207}Pb/^{206}Pb, ^{208}Pb/^{232}Th,
          ^{40}Ar/^{39}Ar, ^{40}K/^{40}Ca, ^{187}Os/^{187}Re,
          ^{87}Sr/^{87}Rb, ^{143}Nd/^{144}Nd or ^{176}Hf/^{177}Hf age.

          ‘s[t]’: the standard error of ‘t’

          ‘disp[t]’: the standard error of ‘t’ enhanced by sqrt{mswd}
          (only applicable if ‘ model=1’).

     mswd the mean square of the residuals (a.k.a `reduced Chi-square')
          statistic (omitted if ‘model=2’).

     p.value the p-value of a Chi-square test for linearity (omitted if
          ‘model=2’)

     w the overdispersion term, i.e. a two-element vector with the
          standard deviation of the (assumed) Normally distributed
          geological scatter that underlies the measurements, and its
          standard error (only returned if ‘model=3’).

     ski (only reported if ‘x’ has class ‘PbPb’ and ‘growth’ is ‘TRUE’)
          the intercept(s) of the isochron with the Stacey-Kramers
          mantle evolution curve.

     OR, if ‘x’ has class ‘ThU’:

     par if ‘x$type=1’ or ‘x$type=3’: the best fitting
          ^{230}Th/^{232}Th intercept, ^{230}Th/^{238}U slope,
          ^{234}U/^{232}Th intercept and ^{234}U/^{238}U slope, OR, if
          ‘x$type=2’ or ‘x$type=4’: the best fitting ^{234}U/^{238}U
          intercept, ^{230}Th/^{232}Th slope, ^{234}U/^{238}U intercept
          and ^{234}U/^{232}Th slope.

     cov the covariance matrix of ‘par’.

     df the degrees of freedom for the linear fit, i.e. (3n-3) if
          ‘x$format=1’ or ‘x$format=2’, and (2n-2) if ‘x$format=3’ or
          ‘x$format=4’

     a if ‘type=1’: the ^{230}Th/^{232}Th intercept; if ‘type=2’: the
          ^{230}Th/^{238}U intercept; if ‘type=3’: the
          ^{234}Th/^{232}Th intercept; if ‘type=4’: the
          ^{234}Th/^{238}U intercept and its propagated uncertainty.

     b if ‘type=1’: the ^{230}Th/^{238}U slope; if ‘type=2’: the
          ^{230}Th/^{232}Th slope; if ‘type=3’: the ^{234}U/^{238}U
          slope; if ‘type=4’: the ^{234}U/^{232}Th slope and its
          propagated uncertainty.

     cov.ab the covariance between ‘a’ and ‘b’.

     mswd the mean square of the residuals (a.k.a `reduced Chi-square')
          statistic.

     p.value the p-value of a Chi-square test for linearity.

     y0 a three-element vector containing:

          ‘y’: the initial ^{234}U/^{238}U-ratio

          ‘s[y]’: the standard error of ‘y’

          ‘disp[y]’: the standard error of ‘y’ enhanced by sqrt{mswd}.

     age a two (or three) element vector containing:

          ‘t’: the initial ^{234}U/^{238}U-ratio

          ‘s[t]’: the standard error of ‘t’

          ‘disp[t]’: the standard error of ‘t’ enhanced by sqrt{mswd}
          (only reported if ‘model=1’).

     w the overdispersion term, i.e. a two-element vector with the
          standard deviation of the (assumedly) Normally distributed
          geological scatter that underlies the measurements, and its
          standard error.

     d a matrix with the following columns: the X-variable for the
          isochron plot, the analytical uncertainty of X, the
          Y-variable for the isochron plot, the analytical uncertainty
          of Y, and the correlation coefficient between X and Y.

     xlab the x-label of the isochron plot

     ylab the y-label of the isochron plot

     OR if ‘x’ has class ‘UPb’:

     par if ‘model=1’ or ‘2’, a three element vector containing the
          isochron age and the common Pb isotope ratios. If ‘model=3’,
          adds a fourth element with the overdispersion parameter w.

     cov the covariance matrix of ‘par’

     logpar the logarithm of ‘par’

     logcov the logarithm of ‘cov’

     n the number of analyses in the dataset

     df the degrees of freedom for the linear fit, i.e. 2n-3

     a the y-intercept and its standard error

     b the isochron slope and its standard error

     cov.ab the covariance between ‘a’ and ‘b’.

     mswd the mean square of the residuals (a.k.a `reduced Chi-square')
          statistic.

     p.value the p-value of a Chi-square test for linearity.

     y0 a two or three-element vector containing:

          ‘y’: the initial ^{206}Pb/^{204}Pb-ratio (if ‘type=1’ and
          ‘x$format=4,5’ or ‘6’); ^{207}Pb/^{204}Pb-ratio (if ‘type=2’
          and ‘x$format=4,5’ or ‘6’); ^{208}Pb/^{206}Pb-ratio (if
          ‘type=1’ and ‘x$format=7’ or ‘8’); ^{208}Pb/^{207}Pb-ratio
          (if ‘type=2’ and ‘x$format=7’ or ‘8’);
          ^{206}Pb/^{208}Pb-ratio (if ‘type=3’ and ‘x$format=7’ or
          ‘8’); or ^{207}Pb/^{208}Pb-ratio (if ‘type=4’ and
          ‘x$format=7’ or ‘8’).

          ‘s[y]’: the standard error of ‘y’

          ‘disp[y]’: the standard error of ‘y’ enhanced by sqrt{mswd}
          (only returned if ‘model=1’)

     y0label the y-axis label of the isochron plot

     age a two (or three) element vector containing:

          ‘t’: the isochron age

          ‘s[t]’: the standard error of ‘t’

          ‘disp[t]’: the standard error of ‘t’ enhanced by sqrt{mswd}
          (only reported if ‘model=1’).

     xlab the x-label of the isochron plot

     ylab the y-label of the isochron plot

_R_e_f_e_r_e_n_c_e_s:

     Ludwig, K.R. and Titterington, D.M., 1994. Calculation of
     ^{230}Th/U isochrons, ages, and errors. Geochimica et Cosmochimica
     Acta, 58(22), pp.5031-5042.

     Ludwig, K.R., 1998. On the treatment of concordant uranium-lead
     ages. Geochimica et Cosmochimica Acta, 62(4), pp.665-676.

     Nicolaysen, L.O., 1961. Graphic interpretation of discordant age
     measurements on metamorphic rocks. Annals of the New York Academy
     of Sciences, 91(1), pp.198-206.

     Titterington, D.M. and Halliday, A.N., 1979. On the fitting of
     parallel isochrons and the method of maximum likelihood. Chemical
     Geology, 26(3), pp.183-195.

     Vermeesch, P., 2020. Unifying the U-Pb and Th-Pb methods: joint
     isochron regression and common Pb correction, Geochronology, 2,
     119-131.

     York, D., 1966. Least-squares fitting of a straight line. Canadian
     Journal of Physics, 44(5), pp.1079-1086.

     York, D., 1968. Least squares fitting of a straight line with
     correlated errors. Earth and Planetary Science Letters, 5,
     pp.320-324.

     York, D., Evensen, N.M., Martinez, M.L. and De Basebe Delgado, J.,
     2004. Unified equations for the slope, intercept, and standard
     errors of the best straight line. American Journal of Physics,
     72(3), pp.367-375.

_S_e_e _A_l_s_o:

     ‘york’, ‘titterington’, ‘ludwig’

_E_x_a_m_p_l_e_s:

     attach(examples)
     isochron(RbSr)
     
     fit <- isochron(ArAr,inverse=FALSE,plot=FALSE)
     
     dev.new()
     isochron(ThU,type=4)
     

