\begin{refsection}

\chapter{Ar--Ar and K--Ca}
\label{ch:ArArKCa-R}

\section{Ar--Ar}\label{sec:ArAr-R}

\noindent\begin{minipage}[t]{.3\linewidth}
\strut\vspace*{-\baselineskip}\newline
\includegraphics[width=\linewidth]{../figures/PbPbFormats.png}
\end{minipage}
\begin{minipage}[t]{.7\textwidth}
  \texttt{IsoplotR} accommodates three Ar--Ar formats. See
  Section~\ref{sec:ArAr} for details.
\end{minipage}

\begin{console}
ArAr <- read.data('ArAr3.csv',method='Ar-Ar',format=3)
\end{console}

\noindent\begin{minipage}[t]{.15\linewidth}
\strut\vspace*{-\baselineskip}\newline
\includegraphics[width=\linewidth]{../figures/ArArPlotdevices.png}\\
\end{minipage}
\begin{minipage}[t]{.85\textwidth}
  Ar--Ar data can be visualised on six different plot devices.
  Additionally, the single aliquot age estimates can also be reported
  in a downloadable data table.
\end{minipage}

The default atmospheric \textsuperscript{40}Ar/\textsuperscript{36}Ar
ratio is given by \citet{lee2006}, and the \textsuperscript{40}K decay
constant by \citet{renne2011}. These values can be changed here:\\

\noindent\includegraphics[width=.7\linewidth]{../figures/ArArLambda.png}

\begin{script}
# use the atmospheric ratio from Nier (1950):
settings('iratio','40Ar36Ar',295.5,0.5)
\end{script}

\section{K--Ca}\label{sec:KCa-R}

\noindent\begin{minipage}[t]{.3\linewidth}
\strut\vspace*{-\baselineskip}\newline
\includegraphics[width=\linewidth]{../figures/PbPbFormats.png}
\end{minipage}
\noindent\begin{minipage}[t]{.15\linewidth}
\strut\vspace*{-\baselineskip}\newline
\includegraphics[width=\linewidth]{../figures/PbPbPlotdevices.png}\\
\end{minipage}
\begin{minipage}[t]{.55\textwidth}
  \texttt{IsoplotR} accommodates three K--Ca formats (see
  Section~\ref{sec:K-Ca} for details), which can be analysed by the
  same methods as the Ar--Ar method minus the age spectra function.
\end{minipage}

\begin{script}
KCa <- read.data('KCa3.csv',method='K-Ca',format=3)
\end{script}

\noindent\includegraphics[width=.7\linewidth]{../figures/KCaLambda.png}\\

\noindent The default value for the inherited Ca component is given by
\citet{moore1972}. This option is hidden from the \texttt{isochron}
function. The default K decay constant is the same as for the Ar--Ar
method \citep{renne2011}.

\begin{script}
# use the Steiger and Jaeger (1977) values with zero uncertainty
settings('lambda','K40',5.543e-4,0)
\end{script}

\section{Isochrons}\label{sec:ArAr-isochrons-R}

\begin{enumerate}

\item \texttt{IsoplotR} offers the same three options to deal with the
  scatter of the data around the best-fit isochron line as the generic
  regression function of Section~\ref{sec:OtherRegression} and the
  Pb--Pb isochron function (Section~\ref{sec:PbPb-isochrons-R}).

\noindent\begin{minipage}[t]{.45\linewidth}
\strut\vspace*{-\baselineskip}\newline
\includegraphics[width=\linewidth]{../figures/ArArIsochronModels.png}
\end{minipage}
\begin{minipage}[t]{.55\linewidth}
  These three models represent three different ways to capture any
  excess dispersion of the data relative to the nominal uncertainties
  (Figure~\ref{fig:isochronMSWD}).
\end{minipage}

\begin{console}
isochron(ArAr,model=3)
\end{console}

\noindent\begin{minipage}[t]{.45\linewidth}
\strut\vspace*{-\baselineskip}\newline
\includegraphics[width=\linewidth]{../figures/KCaIsochronModels.png}
\end{minipage}
\begin{minipage}[t]{.55\linewidth}
  The same three models are available for K--Ca data as well.
\end{minipage}

\begin{console}
isochron(KCa,model=2)
\end{console}

\item \noindent\begin{minipage}[t]{.22\linewidth}
\strut\vspace*{-\baselineskip}\newline
\includegraphics[width=\linewidth]{../figures/ArArisochronInverse.png}
\end{minipage}
\begin{minipage}[t]{.78\linewidth}
Data can be fitted using conventional
(\textsuperscript{40}Ar/\textsuperscript{36}Ar
vs. \textsuperscript{39}Ar/\textsuperscript{36}Ar) or inverse
(\textsuperscript{36}Ar/\textsuperscript{40}Ar
vs. \textsuperscript{39}Ar/\textsuperscript{40}Ar) isochrons
(Section~\ref{sec:inverseIsochrons}).
\end{minipage}

\begin{console}
isochron(ArAr,inverse=FALSE)
\end{console}

\noindent\begin{minipage}[t]{.22\linewidth}
\strut\vspace*{-\baselineskip}\newline
\includegraphics[width=\linewidth]{../figures/PbPbisochronInverse.png}
\end{minipage}
\begin{minipage}[t]{.78\linewidth}
For the K--Ca method, conventional isochrons plot
\textsuperscript{40}Ca/\textsuperscript{44}Ca
vs. \textsuperscript{40}K/\textsuperscript{44}Ca, and inverse plot
\textsuperscript{44}Ca/\textsuperscript{40}Ca
vs. \textsuperscript{40}K/\textsuperscript{40}Ca.
\end{minipage}

\begin{console}
isochron(KCa,inverse=TRUE)
\end{console}

\item The appearance and numerical behaviour of Ar--Ar and K--Ca
  isochrons can be modified using the same options as the Pb--Pb
  isochrons of Section~\ref{sec:PbPb-isochrons-R}, and the general
  regression of Section~\ref{sec:OtherRegression}.

\noindent\begin{minipage}[t]{.45\linewidth}
\strut\vspace*{-\baselineskip}\newline
\includegraphics[width=\linewidth]{../figures/KCaIsochronOtherOptions.png}
\end{minipage}
\begin{minipage}[t]{.55\linewidth}
Tick the checkbox to propagate decay constant uncertainties
(\texttt{exterr}) and label the error ellipses with the row numbers of
the input data (\texttt{show.numbers}), use the textboxes to set the
axis limits (\texttt{xlim} and \texttt{ylim}), significance level
(\texttt{alpha}), significant digits (\texttt{sigdig}), the fill and
stroke colour of the error ellipses (\texttt{ellipse.fill} and
\texttt{ellipse.stroke}), font size (\texttt{cex}) and colour legend
(\texttt{levels} and \texttt{clabel}).
\end{minipage}

\begin{script}
isochron(KCa,inverse=TRUE,exterr=FALSE,show.numbers=TRUE,
         xlim=c(0,2),ylim=c(0,0.02),ellipse.fill=rgb(0.5,1,0.5,0.2))
\end{script}
  
\end{enumerate}

\section{Ages}\label{sec:ArArKCaAges}

Like the Pb--Pb and Th--Pb methods of Chapter~\ref{ch:ThPbPb-R}, also
the K--Ca method is usually applied to cogenetic aliquots for isochron
regression. However the Ar--Ar method is also commonly applied to
detrital grains, which may exhibit a wide range of true ages.  For
both both the Ar--Ar and K--Ca methods, \texttt{IsoplotR} produces
data tables of the single aliquot age estimates, which can be further
analysed and displayed as age spectra, and radial, weighted mean, KDE
and CAD plots.

The atmospheric argon correction for the Ar--Ar ages can be specified
in one of two ways:\\

\noindent\begin{minipage}[t]{.5\linewidth}
\strut\vspace*{-\baselineskip}\newline
\includegraphics[width=\linewidth]{../figures/ArAri2i.png}
\end{minipage}
\begin{minipage}[t]{.5\linewidth}
Ticking this box uses the y-intercept of an isochron fit through all
the Ar data as an initial
\textsuperscript{40}Ar/\textsuperscript{36}Ar-ratio for the age
calculations. Unticking it uses the atmospheric ratio specified above.
\end{minipage}

\begin{script}
# i2i stands for 'intercept to initial'
age(ArAr,i2i=TRUE) 
\end{script}

\noindent\begin{minipage}[t]{.5\linewidth}
\strut\vspace*{-\baselineskip}\newline
\includegraphics[width=\linewidth]{../figures/KCai2i.png}
\end{minipage}
\begin{minipage}[t]{.5\linewidth}
Similarly, for the K--Ca method, the
\textsuperscript{40}Ca/\textsuperscript{44}Ca-ratio of the inherited
Ca can also be determined from a common isochron.
\end{minipage}

\begin{console}
age(KCa,i2i=TRUE)
\end{console}

\noindent\begin{minipage}[t]{.55\linewidth}
\strut\vspace*{-\baselineskip}\newline
\includegraphics[width=\linewidth]{../figures/KCaNominalInitials.png}
\end{minipage}
\begin{minipage}[t]{.45\linewidth}
Or alternatively, the
\textsuperscript{40}Ca/\textsuperscript{44}Ca-ratio can also be
retrieved from \verb|settings('iratio','Ca40Ca44')|.
\end{minipage}

\begin{console}
settings('iratio','Ca40Ca44',50,0)
age(KCa)
\end{console}

\section{Ar--Ar age spectra}\label{sec:ArArAgeSpectra}

There are just two minor differences between Ar--Ar age spectra and
the generic age spectrum function of
Section~\ref{sec:OtherAgeSpectra}. First, an atmospheric argon
correction can be made to each aliquot in exactly the same way as
described in Section~\ref{sec:ArArKCaAges}.

\begin{console}
agespectrum(ArAr,plateau=TRUE,i2i=TRUE)
\end{console}

\noindent\begin{minipage}[t]{.35\linewidth}
\strut\vspace*{-\baselineskip}\newline
\includegraphics[width=\linewidth]{../figures/ArArExterr.png}
\end{minipage}
\begin{minipage}[t]{.65\linewidth}
Second, the \textsuperscript{40}Ar decay constant uncertainty can be
propagated into the plateau age.
\end{minipage}

\begin{console}
agespectrum(ArAr,plateau=TRUE,exterr=TRUE)
\end{console}

The remaining options are the same as the generic age spectrum
function:\\

\noindent\begin{minipage}[t]{.4\linewidth}
\strut\vspace*{-\baselineskip}\newline
\includegraphics[width=\linewidth]{../figures/ArArAgeSpectrumOtherOptions.png}\\
\end{minipage}
\begin{minipage}[t]{.6\linewidth}
  If requested, the plateau age can be computed either using the
  ordinary weighted mean algorithm of Equation~\ref{eq:wtdmean}, or
  the random effects model of Equation~\ref{eq:wtdmean-model-3}.  The
  rectangular segments of the age spectrum can be coloured based on
  any additional parameter. Here a topographic colour ramp is used for
  aliquots that belong to the age plateau, whereas segments that do
  not belong to the plateau are left empty. The colour label uses an
  \texttt{R}-expression to generate superscripts to typeset the
  \textsuperscript{40}Ar/\textsuperscript{36}Ar ratio.
\end{minipage}

The \textsuperscript{40}Ar/\textsuperscript{36}Ar-ratio that forms the
basis of the colour scale can be extracted from the \texttt{ArAr} data
object in this example:

\begin{script}
agespectrum(ArAr,levels=1/ArAr$x[,'Ar36Ar40'],plateau=TRUE,
            plateau.col=topo.colors(n=100,alpha=0.5),
            exterr=TRUE,i2i=FALSE,random.effects=FALSE,
            clabel=expression(''^40*'Ar/'^36*'Ar'))
\end{script}
  
\section{Radial, weighted mean, KDE and CAD plots}
\label{sec:ArArKCaOtherPlots}

The settings for the radial, weighted mean, KDE and CAD plots combine
the generic settings for those graphical devices
(Sections~\ref{sec:OtherRadial}--\ref{sec:OtherCAD}) with the settings
for the age calculator.\\

\noindent\begin{minipage}[t]{.45\linewidth}
\strut\vspace*{-\baselineskip}\newline
\includegraphics[width=\linewidth]{../figures/PbPbRadialOptions.png}
\end{minipage}
\begin{minipage}[t]{.55\linewidth}
  For example, here are shown the GUI settings for a radial plot of
  numbered Pb--Pb data, which is plotted on a linear scale that
  stretches from 4555 to 4575~Ma and is centred around 4565~Ma. Single
  grain ages are corrected using the isochron based common Pb
  composition, and a single age component is fitted to them. The
  standard errors of the central age and the single component age
  include decay constant uncertainties.
\end{minipage}

\begin{console}
radialplot(PbPb,show.numbers=TRUE,exterr=TRUE,transformation='linear',
           k=1,from=4555,to=4575,z0=4565)
\end{console}

Similarly, the weighted mean, KDE and CAD functions work exactly as
the generic versions of Chapter~\ref{ch:generic-R}, with the only
difference being the common Pb correction, and the ability to
propagate external uncertainties.\\

\noindent CLI examples:

\begin{enumerate}

\item The weighted mean using a nominal common Pb correction with
  \textsuperscript{206}Pb/\textsuperscript{204}Pb = 9.15 and
  \textsuperscript{207}Pb/\textsuperscript{204}Pb = 10.23; applying
  the random effects model and plotting the ranked ages:
  
\begin{script}
settings('iratio','Pb206Pb204',9.15)
settings('iratio','Pb206Pb204',10.23)
weightedmean(PbPb,common.Pb=1,random.effects=TRUE,ranked=TRUE)
\end{script}

\item An orange KDE without histogram or rug plot, using an isochron
  based common Pb correction, with a 20~Myr bandwidth and axis limits
  from 4500 to 4600~Ga:

\begin{console}
kde(PbPb,kde.col='orange',rug=FALSE,show.hist=FALSE,
    common.Pb=2,bw=20,from=4500,to=4600)
\end{console}

\item A CAD with without vertical lines and steps marked by `x':

\begin{console}
cad(PbPb,verticals=TRUE,pch='x')
\end{console}
  
\end{enumerate}

\printbibliography[heading=subbibliography]

\end{refsection}
