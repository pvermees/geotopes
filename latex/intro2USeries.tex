\chapter[U-series dating]{U-series disequilibrium methods}
\label{ch:intro2Useries}

In Section \ref{sec:decay-series}, we saw that the
$^{235}$U-$^{207}$Pb and $^{238}$U-$^{206}$Pb decay series generally
reach a state of \emph{secular equilibrium}, in which the activity
(expressed in decay events per unit time) of each intermediate
daughter product is the same, so that:

$$D_n \lambda_n = \cdots  = D_2 \lambda_2 = D_1 \lambda_1 = P \lambda_P$$

as described by Equation \ref{eq:NDnLn}. However, certain natural
processes can disturb this equilibrium situation, such as chemical
weathering, precipitation from a solution, (re-)crystallisation
etc. The leads to two new types of chronometric systems:

\begin{enumerate}
\item An intermediate daughter isotope in the decay series is
  separated from its parent nuclide incorporated into a rock or
  sediment, and decays according to its own half life.
\item A parent nuclide has separated itself from its previous decay
  products and it takes some time for secular equilibrium to be
  re-established.
\end{enumerate}

This idea is most frequently applied to the $^{238}$U-decay series,
notably $^{230}$Th and $^{234}$U. The first type of disequilibrium
dating forms the basis of the $^{234}$U-$^{238}$U and $^{230}$Th
methods (Sections \ref{sec:234238} and \ref{sec:230}). The second
forms the basis of the $^{230}$Th-$^{238}$U method (Section
\ref{sec:230238})

\section{The $^{234}$U-$^{238}$U method}
\label{sec:234238}

The activity ratio of $^{238}$U to its third radioactive daughter
$^{234}$U in the world's oceans is $A(^{234}U)/A(^{238}U)$ $\equiv$
$\gamma_\circ$ $\approx$ 1.15. The slight enrichment of the $^{234}$U
over $^{238}$U is attributed to $\alpha$-recoil of its immediate
parent $^{234}$Th and the fact that $^{234}$U is more `loosely bound'
inside the crystal lattice of the host mineral, because it is
preferentially seated in sites which have undergone radiation
damage. Once the oceanic U is incorporated into the crystal structure
of marine carbonates, the radioactive equilibrium gradually restores
itself with time. The total activity of $^{234}$U is made up of a
component which is supported by secular equilibrium (and equals the
activity of $^{238}U$) and an `excess' component, which decays with
time:

\begin{equation}
A(^{234}U) = A(^{238}U) + A(^{234}U)^x_\circ e^{-\lambda_{234}t} 
\label{eq:A234}
\end{equation}

where $A(^{234}U)^x_\circ$ is the initial amount of excess $^{234}$U
and $\lambda_{234}$ = 2.8234 $\times$ 10$^{-6}$ yr$^{-1}$ (t$_{1/2}$ =
245.5 kyr). Let $A(^{234}U)_\circ$ be the initial total $^{234}$U
activity. Then:

\begin{equation}
A(^{234}U) = A(^{238}U) + \left[A(^{234}U)_\circ - A(^{238}U) \right] e^{-\lambda_{234}t} 
\label{eq:A234b}
\end{equation}

Dividing by A($^{238}$U):

\begin{equation}
\frac{A(^{234}U)}{A(^{238}U)} = 1 + [ \gamma_\circ - 1 ] e^{-\lambda_{234}t} 
\label{eq:A234A238}
\end{equation}

Which can be solved for $t$ until about 1 Ma.

\section{The $^{230}$Th method}
\label{sec:230}

U and Th are strongly incompatible elements. This causes chemical
fractionation and disturbs the secular equilibrium of the $^{238}$U
decay series in young volcanic rocks. It is commonly found that the
activity ratio $A(^{230}Th)/A(^{238}U)$ $>$ 1. As expected, the
secular equilibrium between $^{234}$U and $^{238}$U is not disturbed
by chemical fractionation, so that $A(^{234}U)/A(^{238}U)$ = 1. The
total $^{230}$Th activity is given by:

\begin{equation}
A(^{230}Th) = A(^{230}Th)_\circ e^{-\lambda_{230}t} + A(^{238}U)(1-e^{-\lambda_{230}t})
\label{eq:A230}
\end{equation}

where $A(^{230}Th)_\circ$ is the initial amount of $^{230}$Th at the
time of crystallisation and $A(^{238}U)=A(^{234}U)$ due to secular
equilibrium of the U isotopes. Thus, the first term of Equation
\ref{eq:A230} increases with time from 0 to A($^{238}$U) while the
second term decreases from $A(^{230}Th)_\circ$ to 0. Dividing by
$A(^{232}Th)$ yields a linear relationship between
$A(^{230}Th)/A(^{232}Th)$ and $A(^{238}U) / A(^{232}Th)$:

\begin{equation}
\frac{A(^{230}Th)}{A(^{232}Th)} = \frac{A(^{230}Th)_\circ}{A(^{232}Th)} e^{-\lambda_{230}t} + 
\frac{A(^{238}U)}{A(^{232}Th)}(1-e^{-\lambda_{230}t})
\label{eq:230232}
\end{equation}

This forms an isochron with slope ($1-e^{-\lambda_{230}t}$), from
which the age $t$ can be calculated. This method is applicable to
volcanic rocks and pelitic ocean sediments ranging from 3ka to 1Ma.

\section{The $^{230}$Th-U method}
\label{sec:230238}

Uranium is significantly more soluble in water than Th. As a result,
the intermediate daughter $^{230}$Th is largely absent from sea
water. Thus, lacustrine and marine carbonate rocks contain some U but
virtually no Th at the time of formation. The $^{230}$Th activity
increases steadily with time as a result of $^{234}$U decay. The total
$^{230}$Th activity consists of a growing component $A(^{230}Th)^s$
that is in secular equilibrium with $^{238}U$ and a shrinking
component $A(^{230}Th)^x$ of `excess' $^{230}$Th produced by the
surplus of $^{234}$U commonly found in ocean water (see section
\ref{sec:234238}):

\begin{align}
~ & A(^{230}Th) =  A(^{230}Th)^s + A(^{230}Th)^x \label{eq:230total}\\
\mbox{with:~} & A(^{230}Th)^s = A(^{238}U) (1-e^{-\lambda_{230}t}) \label{eq:230s}\\
\mbox{and:~} & A(^{230}Th)^x = \frac{\lambda_{230}}{\lambda_{230}-\lambda_{234}} 
A(^{234}U)_\circ^x\left(e^{-\lambda_{234}t}-e^{-\lambda_{230}t}\right) \label{eq:230x}
\end{align}

In which the expression for $A(^{230}Th)^x$ follows from Equation
\ref{eq:ND1} and $A(^{234}U)_\circ^x$ denotes the initial amount of
excess $^{234}$U activity (as in Section \ref{sec:234238}). Taking
into account that $A(^{234}U)_\circ^x = A(^{234}U)_\circ -
A(^{238}U)$, and dividing by $A(^{238}U)$, we obtain:

\begin{equation}
\frac{A(^{230}Th)^x}{A(^{238}U)} = \frac{\lambda_{230}}{\lambda_{230}-\lambda_{234}} (\gamma_\circ-1)
\left(e^{-\lambda_{234}t}-e^{-\lambda_{230}t}\right)
\label{eq:230238x}
\end{equation}

in which $\gamma_\circ \equiv A(^{234}U)/A(^{238}U)$ as defined in
Section \ref{sec:234238}. The formation age of the carbonate can be
calculated by substituting Equations \ref{eq:230s} and \ref{eq:230238x}
into \ref{eq:230total} and solving for $t$. 

\begin{equation}
  \frac{A(^{230}Th)}{A(^{238}U)} = 1 - e^{-\lambda_{230}t} +
  \frac{\lambda_{230}}{\lambda_{230}-\lambda_{234}} (\gamma_\circ-1)
\left(e^{-\lambda_{234}t}-e^{-\lambda_{230}t}\right)
\label{eq:230238}
\end{equation}

If $\gamma_\circ = 1$ (i.e., the water is in secular equilibrium for
U), then Equation \ref{eq:230total} simplifies to:

\begin{equation}
\frac{A(^{230}Th)}{A(^{238}U)} = 1-e^{-\lambda_{230}t}
\label{eq:230238b}
\end{equation}

If $\gamma_\circ \neq 1$, Equation \ref{eq:230238b} yields ages that
are systematically too old (if $\gamma_\circ > 1$) or too young (if
$\gamma_\circ < 1$).
