\chapter{Introduction}
\label{ch:intro}

The field of \emph{Isotope Geology} investigates the isotopic
composition of major and trace elements contained in rocks and
minerals, with the aim to better understand geological processes.
Isotopic techniques are used to address a wide range of geological
problems, such as the age of the Earth, the origin and formation of
magmatic rocks, palaeotemperatures in sedimentary basins,
palaeoclimatology, etc.  Isotope geochemistry forms an integral part of
modern Earth Sciences and numerous important discoveries have been
made thanks to this research.  Awareness of these techniques is
required to understand research reports and geological interpretations
based on isotopic methods. Isotope geochemistry plays an important
role in peripheral fields of research such as planetology (origin and
evolution of the Solar system) and archaeology (origin and age of
settlements, tools and artifacts).\\

The use of naturally occurring radioactive isotopes to date minerals
and rocks is the oldest branch of isotope geology. The foundations of
these so-called isotopic or radiometric dating methods were laid
shortly after the turn of the XX\textsuperscript{th} century with the
discovery of the laws of radioactive decay by eminent physicists such
as Ernest Rutherford and Frederick Soddy \citep{rutherford1902a,
  rutherford1902b}.  The application of these principles to the field
of Geology and the calibration of the geological time scale were
pioneered by Arthur \citet{holmes1911, holmes1913,
  holmes1947}. Initially, radiometric geochronology was exclusively
based on uranium and its daughter products, but with the development
of increasingly sensitive analytical equipment, ever more isotopic
`clocks' were added over the course of the century: Rb/Sr
\citep{hahn1943}, $^{14}$C \citep{libby1946}, K/Ar
\citep{aldrich1948}, $^{238}$U fission tracks \citep{price1963},
$^{40}$Ar/$^{39}$Ar \citep{merrihue1966}, Sm/Nd \citep{lugmair1974},
etc.\\

During the 1960s, geochemists began to investigate the non-radiogenic
composition of igneous rocks with the aim to understand their source
and origin. This line of research greatly expanded over the course of
the 1970s and 80s and nowadays the isotopic composition of elements
such as Sr and Nd in rocks and minerals is an established petrogenetic
indicator. The discovery that the isotopes of the light elements (H,
C, N, O, S) are fractionated by physical and chemical processes dates
back to the 1930s. The isotopic composition of these elements can
therefore be used to detect and understand the hydrospheric and
lithospheric processes causing such fractionation
\citep{urey1947}. This has led to a better understanding of the
physiochemical conditions under which rocks and minerals are
formed. Temperature is the most important of these parameters and the
aforementioned elements are often used for palaeothermometry.\\

These lecture notes cover the first half of an Isotope Geology module
at UCL that deals with the geochronological aspects of the subject.
The second part of the module deals with stable isotopes. It is taught
by Dr.~Susan Little and covered in a separate set of notes. The core
of the geochronology notes is formed by Prof. Peter van den Haute's
lecture notes (in Dutch) at the University of Ghent.  This was
expanded with additional material, notably on the subjects of
cosmogenic nuclide geochronology (Chapter~\ref{sec:cosmo}) and U-Th-He
dating (Section~\ref{sec:U-Th-He}). Some figures were modified from
published sources, including \citet{allegre2008}, \citet{braun2006},
and \citet{galbraith2005}. These books are recommended further reading
material, as is the detailed textbook by \citet{dickin2005}, from
which both \citet{allegre2008} and van den Haute heavily
borrowed. Additional lecture material, including the data files used
in the programming practicals of Chapter~\ref{sec:programming}, can be
found at \texttt{https://github.com/pvermees/geotopes/}.

\begin{thebibliography}{16}
\providecommand{\natexlab}[1]{#1}
\providecommand{\url}[1]{\texttt{#1}}
\expandafter\ifx\csname urlstyle\endcsname\relax
  \providecommand{\doi}[1]{doi: #1}\else
  \providecommand{\doi}{doi: \begingroup \urlstyle{rm}\Url}\fi

\bibitem[Aldrich and Nier(1948)]{aldrich1948}
Aldrich, L.~T. and Nier, A.~O.
\newblock Argon 40 in potassium minerals.
\newblock \emph{Physical Review}, 74\penalty0 (8):\penalty0 876, 1948.

\bibitem[All{\`e}gre(2008)]{allegre2008}
All{\`e}gre, C.~J.
\newblock \emph{Isotope geology}.
\newblock Cambridge University Press, 2008.

\bibitem[Braun et~al.(2006)Braun, Van Der~Beek, and Batt]{braun2006}
Braun, J., Van Der~Beek, P., and Batt, G.
\newblock \emph{Quantitative thermochronology: numerical methods for the
  interpretation of thermochronological data}.
\newblock Cambridge University Press, 2006.

\bibitem[Dickin(2005)]{dickin2005}
Dickin, A.~P.
\newblock \emph{Radiogenic isotope geology}.
\newblock Cambridge University Press, 2005.

\bibitem[Galbraith(2005)]{galbraith2005}
Galbraith, R.~F.
\newblock \emph{Statistics for fission track analysis}.
\newblock CRC Press, 2005.

\bibitem[Hahn et~al.(1943)Hahn, Strassman, Mattauch, and Ewald]{hahn1943}
Hahn, O., Strassman, F., Mattauch, J., and Ewald, H.
\newblock {Geologische Altersbestimmungen mit der strontiummethode}.
\newblock \emph{Chem. Zeitung}, 67:\penalty0 55--6, 1943.

\bibitem[Holmes(1911)]{holmes1911}
Holmes, A.
\newblock The association of lead with uranium in rock-minerals, and its
  application to the measurement of geological time.
\newblock \emph{Proceedings of the Royal Society of London. Series A,
  Containing Papers of a Mathematical and Physical Character}, 85\penalty0
  (578):\penalty0 248--256, 1911.

\bibitem[Holmes(1913)]{holmes1913}
Holmes, A.
\newblock \emph{The age of the Earth}.
\newblock Harper \& Brothers, 1913.

\bibitem[Holmes(1947)]{holmes1947}
Holmes, A.
\newblock {The Construction of a Geological Time-Scale}.
\newblock \emph{Transactions of the Geological Society of Glasgow}, 21\penalty0
  (1):\penalty0 117--152, 1947.

\bibitem[Libby(1946)]{libby1946}
Libby, W.~F.
\newblock Atmospheric helium three and radiocarbon from cosmic radiation.
\newblock \emph{Physical Review}, 69\penalty0 (11-12):\penalty0 671, 1946.

\bibitem[Lugmair(1974)]{lugmair1974}
Lugmair, G.
\newblock Sm-Nd ages: a new dating method.
\newblock \emph{Meteoritics}, 9:\penalty0 369, 1974.

\bibitem[Merrihue and Turner(1966)]{merrihue1966}
Merrihue, C. and Turner, G.
\newblock Potassium-argon dating by activation with fast neutrons.
\newblock \emph{Journal of Geophysical Research}, 71\penalty0 (11):\penalty0
  2852--2857, 1966.

\bibitem[Price and Walker(1963)]{price1963}
Price, P. and Walker, R.
\newblock Fossil tracks of charged particles in mica and the age of minerals.
\newblock \emph{Journal of Geophysical Research}, 68\penalty0 (16):\penalty0
  4847--4862, 1963.

\bibitem[Rutherford and Soddy(1902{\natexlab{a}})]{rutherford1902a}
Rutherford, E. and Soddy, F.
\newblock The cause and nature of radioactivity -- part i.
\newblock \emph{The London, Edinburgh, and Dublin Philosophical Magazine and
  Journal of Science}, 4\penalty0 (21):\penalty0 370--396, 1902{\natexlab{a}}.

\bibitem[Rutherford and Soddy(1902{\natexlab{b}})]{rutherford1902b}
Rutherford, E. and Soddy, F.
\newblock The cause and nature of radioactivity -- part ii.
\newblock \emph{The London, Edinburgh, and Dublin Philosophical Magazine and
  Journal of Science}, 4\penalty0 (23):\penalty0 569--585, 1902{\natexlab{b}}.

\bibitem[Urey(1947)]{urey1947}
Urey, H.~C.
\newblock The thermodynamic properties of isotopic substances.
\newblock \emph{Journal of the Chemical Society (Resumed)}, pages 562--581,
  1947.

\end{thebibliography}


%\bibliographystyle{/home/pvermees/Dropbox/abbrvplainnat}
%\bibliography{/home/pvermees/Dropbox/biblio}
