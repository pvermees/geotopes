\documentclass[11pt,openany]{book}

\usepackage[top=3cm,bottom=3cm,left=3cm,right=2cm,twoside,a4paper]{geometry}
%\usepackage[top=2cm,bottom=2cm,left=2cm,right=1.5cm,twoside,a5paper]{geometry}
\usepackage[hidelinks,linktocpage=true]{hyperref}
\usepackage{amsmath,graphicx,wasysym,color,upquote,caption,enumitem}
\setlist[enumerate]{leftmargin=1pc}
\usepackage[style=authoryear,natbib=true,backend=bibtex,giveninits=true]{biblatex}
\addbibresource{/home/pvermees/Dropbox/biblio.bib}

\usepackage{fancyvrb}
\newenvironment{console}
               {\VerbatimEnvironment\begin{Verbatim}[frame=single,framerule=0.4mm]}
               {\end{Verbatim}}

\DefineVerbatimEnvironment{script}
                          {Verbatim}
                          {frame=single,numbers=left,framerule=0.4mm}
                          
\newif\ifpdf
\pdftrue

\newif\ifuclnotes
\uclnotesfalse

\newif\iftraining
\trainingtrue

\raggedbottom

\begin{document}
Like argon, helium is a noble gas that is lost to the environment (and
eventually to space) at high temperatures by volume diffusion.
Additional complication is added by the physical separation of the
parent and daughter nuclides in the U-Th-(Sm)-He system. This
separation results from the energy released during $\alpha$-decay,
which displaces the $\alpha$-particles by up to 16~$\mu$m and may
result in the ejection of helium produced by parent atoms that are
sited near the edges of the host mineral. That lost helium must be
taken into account when interpreting the thermal history of a
sample. For rapidly cooled samples, this can be done by applying a
geometric correction to the U, Th and Sm-measurements. For a sphere:
\begin{equation}
  F_T = 1 - \frac{3}{4}\frac{S}{R} + \frac{1}{16} \left[\frac{S}{R}\right^3
    \label{eq:FTsphere}
\end{equation}

\noindent where $F_T$ is the fraction of helium that is retained in
the grain, $r$ are the is the radius of a sphere with equivalent
surface-to-volume ratio as the mineral habit of interest, and $S$ is
the $alpha$-stopping distance:

\begin{center}
\begin{tabular}{cccccc}
  mineral & \textsuperscript{238}U & \textsuperscript{235}U
  & \textsuperscript{232}Th & \textsuperscript{147}Sm \\ \hline
  apatite & 18.81 & 21.80 & 22.25 & 5.93 \\
  zircon & 15.55 & 18.05 & 18.43 & 4.76 \\
  sphene & 17.46 & 20.25 & 20.68 & 5.47
\end{tabular}
\captionof{table}{Stopping distances ($S$) of $\alpha$-particles in
  apatite, zircon and sphene.}
\label{tab:stoppingdistances}
\end{center}

Most minerals are not spherical but elongated prismatic, and can be
approximated to a first degree as cylinders with radius $r$ and height
$h$:
\begin{equation}
  F_T = 1 - \frac{1}{2}\frac{(r+h)S}{rh} +
  0.2122 \frac{S^2}{rh} + 0.0153 \frac{S^3}{r^3}
  \label{eq:FTcylinder}
\end{equation}

For an extensive list of formulas for even more realistic geometric
shapes, see \citet{ketcham2011}. An $\alpha$-ejection correction can
be applied in one of three ways:

\begin{enumerate}
\item For young samples, the ejected helium can be mathematically
  replaced to a good approximation by dividing the uncorrected
  U--Th--He age by $F_T$ \citet{farley2002}:
  \begin{equation}
    t' = \frac{t}{ (1.04 + 0.245)(U/Th) F_T^{238} + ()}
  \end{equation}
\end{enumerate}

\texttt{IsoplotR} assumes
that such an `$\alpha$-ejection correction' has been applied to the
data \textbf{prior} to age calculation.\\


\end{document}
