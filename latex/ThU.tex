\begin{refsection}

\chapter{\textsuperscript{230}Th--U dating}\label{ch:ThU}

The general principles of U-series disequilibrium dating were laid out
in Chapter~\ref{ch:intro2Useries}. Recall that the radioactive decay
of \textsuperscript{238}U to \textsuperscript{206}Pb produces
short-lived \textsuperscript{230}Th (t\textsubscript{1/2}=76~kyr) and
\textsuperscript{234}U (t\textsubscript{1/2}=246~kyr), which may
fractionate by one of two mechanisms:

\begin{enumerate}
\item U and Th have contrasting chemical properties and are easily
  fractionated during chemical processes such as crystallisation. This
  fractionation disrupts any pre-existing state of secular equilibrium
  between \textsuperscript{230}Th and its parent nuclide
  \textsuperscript{234}U.
\item Preferential leaching of weakly sited \textsuperscript{234}U due
  to energetic recoil of the U-nucleus during
  \textsuperscript{238}U-decay enriches \textsuperscript{234}U
  relative to \textsuperscript{238}U in river and sea water.
\item Redox processes and repeated precipitation--dissolution cycles
  may fractionate \textsuperscript{234}U from \textsuperscript{238}U,
  especially in the presence of organic acids in soils. This can
  produce ground waters that are highly enriched in
  \textsuperscript{234}U relative to \textsuperscript{238}U.
\end{enumerate}

Section~\ref{sec:234238} showed that the fractionation between
\textsuperscript{234}U and \textsuperscript{238}U can be used to date
marine carbonates; Section~\ref{sec:230} showed that the fractionation
between \textsuperscript{230}Th and \textsuperscript{234}U can be used
to date young volcanic rocks; and Section~\ref{sec:230238} combined
the two equations together in a single equation. Recalling
Equation~\ref{eq:230238}
\begin{equation}
  \frac{A(^{230}Th)}{A(^{238}U)} = 1 - e^{-\lambda_{230}t} +
  \frac{\lambda_{230}}{\lambda_{230}-\lambda_{234}} (\gamma_\circ-1)
\left(e^{-\lambda_{234}t}-e^{-\lambda_{230}t}\right)
\end{equation}

\noindent where $\gamma_\circ$ is the oceanic
\textsuperscript{234}U/\textsuperscript{238}U activity ratio.
Equation~\ref{eq:230238} requires that $\gamma_\circ$ is known and
produces non-unique age solutions when
A(\textsuperscript{230}Th)/A(\textsuperscript{238}U)$>1$. Both of
these limitations can be avoided by recasting the equation in the
following form:
\begin{equation}
  \frac{A[{}^{230}Th] - A[{}^{230}Th]_\circ}{A[{}^{238}U]} =
  1 - e^{\lambda_{230}t} -
  \left(\frac{A[{}^{234}U]}{A[{}^{238}U]}-1\right)
  \left(\frac{\lambda_{230}}{\lambda_{234}-\lambda_{230}}\right)
  \left(1-e^{[\lambda_{234}-\lambda_{230}]t}\right)
  \label{eq:Th-U}
\end{equation}

\noindent where $A[\ast]$ is the activity of $\ast$ and
$A[{}^{230}Th]_\circ$ is the `detrital' \textsuperscript{230}Th
component, i.e. the \textsuperscript{230}Th that was already present
in the sample at the time of its formation \citep{kaufman1965,
  ludwig2003b}. This component is unknown but can be estimated by
isochron regression using long-lived \textsuperscript{232}Th as a
normalising factor.\\

\texttt{IsoplotR} uses Equation~\ref{eq:Th-U} as the basis of all its
U-series dating applications.

\section{Data formats}

\section{Isochrons}

\section{Th--U evolution diagrams}

For igneous samples, in which $A[{}^{234}U]/A[{}^{238}U] = 1$, the
second term on the right-hand side of Equation~\ref{eq:Th-U} vanishes
and we can write:
\begin{equation}
  \left(\frac{A[{}^{230}Th]}{A[{}^{232}Th]}\right)_i = 
  \left(\frac{A[{}^{230}Th]}{A[{}^{232}Th]}\right)_\circ +
  \left(\frac{A[{}^{238}U]}{A[{}^{232}Th]}\right)_i
  \left(1-e^{-\lambda_{230}t}\right)
  \label{eq:Th-U-volcanic}
\end{equation}

\noindent for $1 \leq i \leq n$, which can be solved for $t$ and
$\left(A[{}^{230}Th]/A[{}^{232}Th]\right)_\circ$ using the least
squares method of \citet{york2004}. Equation~\ref{eq:Th-U-volcanic}
forms a `Rosholt'-type isochron, which is akin to a `normal' isochron
in Rb-Sr or Ar-Ar geochronology \citep{rosholt1976}. Using
$A[{}^{238}U]$ as the normalising factor instead yields an
`Osmond'-type isochron, which is akin to an `inverse' isochron in
Ar-Ar or Pb-Pb geochronology \citep{osmond1970, ludwig2003b}.\\

For carbonate samples, in which \textsuperscript{234}U and
\textsuperscript{238}U generally are not in secular equilibrium, three
activity ratios are needed to determine the detrital
\textsuperscript{230}Th (and initial \textsuperscript{234}U)
component. This in turn requires three dimensional isochron regression
of the \textsuperscript{230}Th/\textsuperscript{238}U-,
\textsuperscript{232}Th/\textsuperscript{238}U- and
\textsuperscript{234}U/\textsuperscript{238}U-activity
ratios. \texttt{IsoplotR} performs this calculation with the maximum
likelihood algorithm of \citet{ludwig1994}.\\

In addition to (Rosholt and Osmond) isochrons and the usual weighted
mean, radial, CAD and KDE plots, U-series data can also be visualised
on Th-U evolution diagrams.  For carbonate data, these consist of a
scatter plot that sets out the
\textsuperscript{234}U/\textsuperscript{238}U-activity ratios against
the \textsuperscript{230}Th/\textsuperscript{238}U-activity ratios as
error ellipses, and displays the initial
\textsuperscript{234}U/\textsuperscript{238}U-activity ratios and ages
as a set of intersecting lines (Figure~\ref{fig:2}.b).\\

The Th-U evolution diagram has a similar purpose and appearance as the
U-Pb concordia diagram, which also displays compositions and dates
simultaneously. An alternative way of doing so for carbonate samples
is by plotting the initial
\textsuperscript{234}U/\textsuperscript{238}U-ratios against the
\textsuperscript{230}Th-\textsuperscript{234}U-\textsuperscript{238}U-ages
(Figure~\ref{fig:2}.c).  In both types of evolution diagrams,
\texttt{IsoplotR} provides the option to project the raw measurements
along the best fitting isochron line and thereby remove the detrital
\textsuperscript{230}Th-component. This procedure allows a visual
assessment of the degree of homogeneity within a dataset, as is
quantified by the MSWD.\\

Neither the U-series evolution diagram nor the
\textsuperscript{234}U/\textsuperscript{238}U vs. age plot is
applicable to igneous datasets, in which \textsuperscript{234}U and
\textsuperscript{238}U are in secular equilibrium.  For such datasets,
\texttt{IsoplotR} produces an Osmond-style regression plot that is
decorated with a fanning set of isochron lines.


\printbibliography[heading=subbibliography]

\end{refsection}
