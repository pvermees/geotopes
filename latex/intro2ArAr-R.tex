\section{$^{40}$Ar/$^{39}$Ar data reduction}
\label{sec:Ar-Ar-R}

In this exercise, we will reduce some synthetic $^{40}$Ar/$^{39}$Ar
data. You are provided with three input files:
\begin{enumerate}
\item{\tt smpl.csv}: $^{36}$Ar, $^{39}$Ar and $^{40}$Ar as a function
  of time (t) for the sample.
\item{\tt stnd.csv}: the same data for the standard, which is a Fish
  Canyon sanidine with a conventional K-Ar age of 27.8 Ma.
\item{\tt blnk.csv}: a `blank' run, i.e. a measurement of the
  background levels of Argon present in the mass spectrometer in the
  absence of a sample.
\end{enumerate}

To perform the data reduction, please follow the following steps:

\begin{enumerate}
\item Load the three input files.
\item Plot the $^{40}$Ar signal of the sample against time. Do the
  same for the $^{36}$Ar signal in the blank. What is the difference?
\item Perform a linear regression of the $^{36}$Ar, $^{39}$Ar and
  $^{40}$Ar signals through time and determine the intercept at t=0.
\item Subtract the `time zero' intercepts of the blank from those of
  the sample and standard.
\item Apply an atmospheric correction assuming that all $^{36}$Ar has
  an atmospheric origin.
\item Calculate the J-value of the standard.
\item Calculate the age of the sample.
\end{enumerate}
