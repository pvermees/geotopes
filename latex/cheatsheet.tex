\documentclass{article}
\usepackage[subtle]{savetrees}
\usepackage[a4paper, landscape, margin=1.2cm]{geometry}
\usepackage{amsmath,graphicx,caption,color,upquote,enumitem,multicol}
%\usepackage{cmbright} % switch to sans-serif font
\pagestyle{empty}
\title{Key equations}
\author{\vspace{-5ex}}
\date{\vspace{-5ex}}
\setlength\parindent{0pt}
\begin{document}

\begin{multicols}{3}

\section{Constants}

\subsection{Molar mass}

H: 1.00794, He: 4.002602, C: 12.0107, O: 15.9994, Al: 26.98154, Si:
28.0855, K: 39.0983, Ar: 39.948, Ca: 40.078, Rb: 85.4678, Sr: 87.62,
I: 126.904, Xe: 131.293 Nd: 144.242, Sm: 150.36, Eu:151.964, Lu:
174.967, Hf: 178.49, W: 183.84, Re: 186.207 Os: 190.23, Pb: 207.2, Th:
232.04, U: 238.04

\subsection{Isotopic ratios}

\begin{tabular}{p{.03\linewidth}p{.9\linewidth}}
K: & \textsuperscript{40}K/K = 0.01167\% \\

Ar: & \textsuperscript{40}Ar/\textsuperscript{36}Ar (atm) = 298.5;
\textsuperscript{38}Ar/\textsuperscript{36}Ar = 0.187 \\

Rb: & \textsuperscript{85}Rb/\textsuperscript{87}Rb=2.59337 \\

Sr: & \textsuperscript{84}Sr/\textsuperscript{86}Sr=0.056584;
\textsuperscript{86}Sr/\textsuperscript{88}Sr=0.1194 \\

Sm: & \textsuperscript{147}Sm/Sm = 14.99\% \\

U: & \textsuperscript{238}U/\textsuperscript{235}U = 137.818; \\
~ & A($^{234}$U)/A($^{238}$U)[modern ocean] $\approx$ 1.15.

\end{tabular}

\subsection{Half-lives and decay constants}

\begin{tabular}{p{.03\linewidth}p{.9\linewidth}}
C: & t\textsubscript{1/2}(\textsuperscript{14}C) = 5730 yr \\

K: & $\lambda_{\beta^-} = 4.962 \times 10^{-10} \mbox{yr}^{-1}$; 
$\lambda_{e} = 0.581 \times 10^{-10} \mbox{yr}^{-1}$;
$t_{1/2}(e+\beta^-)=1.248$ Gyr \\

Rb: & t\textsubscript{1/2}(\textsuperscript{87}Rb) = 48.8 Gyr \\

Sm: & t\textsubscript{1/2}(\textsuperscript{147}Sm) = 106 Gyr \\

U: & t\textsubscript{1/2}(\textsuperscript{238}U) = 4.468 Gyr;
t\textsubscript{1/2}(\textsuperscript{235}U) = 703.8 Myr;
t\textsubscript{1/2}(\textsuperscript{234}U) = 245.5 kyr;
$\lambda_f = 8.46\times{10}^{-17}$yr\textsuperscript{-1}\\

Th: & t\textsubscript{1/2}(\textsuperscript{232}Th) = 14.05 Gyr;
t\textsubscript{1/2}(\textsuperscript{230}Th) = 75.38 kyr \\

\end{tabular}

\section{Basic notions}
\label{ch:basic-notions}

\subsection{Radioactivity}
\label{sec:radioactivity}

The decay equation:
\begin{equation}
\frac{dP}{dt} = -\lambda P
\label{eq:dPdt}
\end{equation}

Its solution:
\begin{equation}
P = P_\circ e^{-\lambda t}
\label{eq:P}
\end{equation}

The fundamental age equation:
\begin{equation}
t = \frac{1}{\lambda} \ln\left(\frac{D*}{P} + 1\right)
\label{eq:t}
\end{equation}

where $D_*$ marks the radiogenic daughter component. Half-life
vs. decay constant:
\begin{equation}
\frac{P_\circ}{2} = P_\circ e^{-\lambda t_{1/2}} 
\Rightarrow t_{1/2} = \frac{\ln(2)}{\lambda}
\label{eq:T12}
\end{equation}

\subsection{Decay series}
\label{sec:decay-series}

\begin{align}
\mbox{for~} P &: dP/dt = -\lambda_P P \label{eq:P1}\\
\mbox{for~} D_1 &: dD_1/dt = \lambda_P P - \lambda_1 D_1 \label{eq:D1}\\
\mbox{for~} D_* &: dD_*/dt = \lambda_1 D_1 \label{eq:D*}
\end{align}

Assuming that $D_1=0$ at $t=0$:
\begin{equation}
D_1 = \frac{\lambda_P}{\lambda_1 - \lambda_P} P_\circ \left[
  e^{-\lambda_P t} - e^{-\lambda_1 t}\right]
\label{eq:ND1}
\end{equation}

Secular equilibrium:
\begin{equation}
D_1 \lambda_1 = P \lambda_P
\label{eq:ND1L1}
\end{equation}

or, equivalently:
\begin{equation}
\frac{P}{D_1} = \frac{t_{1/2}(P)}{t_{1/2}(D_1)}
\label{eq:NPND1}
\end{equation}

\section{Analytical techniques}
\label{ch:analyticaltechniques}

\subsection{Mass spectrometry}
\label{sec:mass-specs}

Kinetic energy of single-charged ion of $m$ a.m.u. in a mass
spectrometer:
\begin{equation}
E = e V = \frac{m v^2}{2}
\label{eq:E}
\end{equation}

With $e = 1.60219 \times 10^{-19}$C and 1 a.m.u. = 1.660538 $\times
10^{-27}$kg. The mass analyser deflects the ions according to the
following equation:
\begin{equation}
H e v = \frac{m v^2}{r}
\label{eq:Hev}
\end{equation}

from which it follows that:
\begin{equation}
r = \frac{1}{H}\sqrt{\frac{2 m V}{e}}
\label{eq:rH}
\end{equation}

\subsection{Isotope dilution}
\label{sec:isotope-dilution}

Will not be part of the exam.

\subsection{Sample-standard bracketing}
\label{sec:bracketing}

Will not be part of the exam.

\section{Simple parent-daughter pairs}
\label{ch:intro2PD}

\subsection{$^{14}$C dating}
\label{sec:14C}

\begin{equation}
\frac{d^{14}C}{dt} = -\lambda_{14} \times {}^{14}C
\label{eq:d14Cdt}
\end{equation}

where $\lambda_{14}$ = 0.120968 kyr$^{-1}$.
\begin{equation}
t = -\frac{1}{\lambda_{14}}
\ln\left[\frac{d{}^{14}C/dt}{(d{}^{14}C/dt)_\circ}\right]
\label{eq:t14C}
\end{equation}

\subsection{The Rb-Sr method}
\label{sec:Rb-Sr}

Ingrowth equation:
\begin{equation}
{}^{87}Sr = {}^{87}Sr_\circ + {}^{87}Rb (e^{\lambda_{87} t} - 1)
\label{eq:87Sr*}
\end{equation}

where $^{87}Sr_\circ$ is the initial $^{87}$Sr content.
$^{87}$Rb/$^{86}$Sr-ratio is calculated as:
\begin{equation}
\frac{^{87}Rb}{^{86}Sr} =
\frac{Rb}{Sr} \frac{Ab(^{87}Rb)
  M(Sr)}{Ab(^{86}Sr) M(Rb)}
\label{eq:87Rb86Sr}
\end{equation}

where $Ab(\cdot)$ signifies `abundance'.

\subsection{Isochrons}
\label{sec:isochrons}

The universal isochron equation:
\begin{equation}
\frac{D}{d} = \left(\frac{D}{d}\right)_\circ + \frac{P}{d} (e^{\lambda_{P} t} - 1)
\label{eq:isochron}
\end{equation}

where $P$ = the parent isotope, $D$ = the daughter isotope, $d$ = a
non-radiogenic sister isotope of the radiogenic daughter. For the
Rb--Sr method:
\begin{equation}
\frac{^{87}Sr}{^{86}Sr} =
\left(\frac{^{87}Sr}{^{86}Sr}\right)_\circ +
\frac{^{87}Rb}{^{86}Sr} (e^{\lambda_{87} t} - 1)
\label{eq:87Sr86Sr}
\end{equation}

\subsection{The Sm-Nd method}
\label{sec:Sm-Nd}

The age equation:
\begin{equation}
^{143}Nd = ^{143}Nd_\circ + {}^{147}Sm (e^{\lambda_{147} t} - 1)
\label{eq:144Nd*}
\end{equation}

Isochron equation:
\begin{equation}
\frac{^{143}Nd}{^{144}Nd} =
\left(\frac{^{143}Nd}{^{144}Nd}\right)_{\circ} +
\frac{^{147}Sm}{^{144}Nd} \left(e^{\lambda_{147}t} -
1\right)
\label{eq:143Nd147Nd}
\end{equation}

\section{The U-Pb system}
\label{sec:U-Pb}

\begin{equation}
\begin{array}{rl}
^{238}U \rightarrow & {}^{206}Pb + 8\alpha + 6\beta + 47\mbox{MeV} \\ 
^{235}U \rightarrow & {}^{207}Pb + 7\alpha + 4\beta + 45\mbox{MeV} \\
^{232}Th \rightarrow & {}^{208}Pb + 6\alpha + 4\beta + 40\mbox{MeV} 
\end{array}
\label{eq:UThdecay}
\end{equation}

\section{The U-(Th-)Pb method}
\label{sec:U-Th-Pb}

Ingrowth equations:
\begin{equation}
  \begin{array}{rl}
    &{}^{206}Pb^* = {}^{238}U \left(e^{\lambda_{238}t} - 1\right)\\ 
    &{}^{207}Pb^* = {}^{235}U \left(e^{\lambda_{235}t} - 1\right)\\ 
    &{}^{208}Pb^* = {}^{232}Th \left(e^{\lambda_{232}t} - 1\right)
  \end{array}
  \label{eq:Pb*}
\end{equation}

where ${}^{20x}$Pb$^*$ is the radiogenic ${}^{20x}$Pb component
(${}^{20x}$Pb = ${}^{20x}$Pb$^*$ + ${}^{20x}$Pb$_\circ$). The
corresponding age equations are:
\begin{equation}
  \begin{array}{rl}
    t_{206} & = \frac{1}{\lambda_{238}}
    \ln \left(\frac{{}^{206}Pb^*}{{}^{238}U}+1\right)\\
    t_{207} & = \frac{1}{\lambda_{235}}
    \ln \left(\frac{{}^{207}Pb^*}{{}^{235}U}+1\right)\\
    t_{208} & = \frac{1}{\lambda_{232}}
    \ln \left(\frac{{}^{208}Pb^*}{{}^{232}Th}+1\right)
  \end{array}
  \label{eq:tPb*}
\end{equation}

with common Pb correction:
\begin{equation}
\begin{array}{c}
  t_{206}=\frac{1}{\lambda_{238}}\ln\left(\frac{\left(\frac{^{206}Pb}{^{204}Pb}\right)-
    \left(\frac{^{206}Pb}{^{204}Pb}\right)_\circ}{\frac{^{238}U}{^{204}Pb}}+1\right)\\
  t_{207}=\frac{1}{\lambda_{235}}\ln\left(\frac{\left(\frac{^{207}Pb}{^{204}Pb}\right)-
    \left(\frac{^{207}Pb}{^{204}Pb}\right)_\circ}{\frac{^{235}U}{^{204}Pb}}+1\right)\\
  t_{208}=\frac{1}{\lambda_{232}}\ln\left(\frac{\left(\frac{^{208}Pb}{^{204}Pb}\right)-
    \left(\frac{^{208}Pb}{^{204}Pb}\right)_\circ}{\frac{^{232}Th}{^{204}Pb}}+1\right)
\end{array}
\label{eq:tPb}
\end{equation}

\subsection{The Pb-Pb method}
\label{sec:Pb-Pb}

Age equation:
\begin{equation}
\frac{^{207}Pb^*}{^{206}Pb^*} = 
\frac{\left(\frac{^{207}Pb}{^{204}Pb}\right)-\left(\frac{^{207}Pb}{^{204}Pb}\right)_\circ}
{\left(\frac{^{206}Pb}{^{204}Pb}\right)-\left(\frac{^{206}Pb}{^{204}Pb}\right)_\circ}
= \frac{1}{137.818} \frac{e^{\lambda_{235}t}-1}{e^{\lambda_{238}t}-1}
\label{eq:PbPb}
\end{equation}

where $^{238}$U/$^{235}$U=137.818. For modern samples:
\begin{equation}
  \left(\frac{^{207}Pb}{^{206}Pb}\right)^*_p =
  \frac{\lambda_{235}}{137.818\lambda_{238}} = 0.04607
\label{eq:commonPb}
\end{equation}

\section{The K--Ar system}
\label{sec:K-Ar}

Ingrowth equation:
\begin{equation}
\begin{array}{rl}
^{40}Ar & = {}^{40}Ar_\circ + {}^{40}Ar^*\\ 
\mbox{where~} {}^{40}Ar^* & =
  \frac{\lambda_e}{\lambda} {}^{40}K \left( e^{\lambda t} - 1 \right)
\end{array}
\label{eq:Ar}
\end{equation}

\subsection{K-Ar dating}

Age equation:
\begin{equation}
t = \frac{1}{\lambda} \ln\left[ 1 + \frac{\lambda}{\lambda_e}
  \left(\frac{^{40}Ar^*}{^{40}K}\right) \right]
\label{eq:K-Ar}
\end{equation}

Isochron equation:
\begin{equation}
\frac{^{40}Ar}{^{36}Ar} = \left(\frac{^{40}Ar}{^{36}Ar}\right)_\circ +
\frac{\lambda_e}{\lambda} \frac{^{40}K}{^{36}Ar} \left( e^{\lambda t} - 1 \right)
\label{eq:K-Ar-isochron}
\end{equation}

\subsection{$^{40}$Ar/$^{39}$Ar dating}
\label{sec:Ar-Ar}

Sample:
\begin{equation}
t_x = \frac{1}{\lambda} \ln\left[
1 + J \left(\frac{^{40}Ar^*}{^{39}Ar}\right)_x 
\right]
\label{eq:Ar-Ar}
\end{equation}

Standard:
\begin{equation}
t_s = \frac{1}{\lambda} \ln\left[
1 + J \left(\frac{^{40}Ar^*}{^{39}Ar}\right)_s 
\right]
\label{eq:J}
\end{equation}

Age equation:
\begin{equation}
\frac{{}^{40}Ar}{{}^{36}Ar} =
\left(\frac{{}^{40}Ar}{{}^{36}Ar}\right)_\circ +
\frac{{}^{39}Ar}{{}^{36}Ar}\frac{e^{\lambda t} - 1}{J}
\label{eq:Ar-Ar-isochron}
\end{equation}

\section{Thermochronology}

\subsection{The U-Th-He method}
\label{sec:U-Th-He}

Ingrowth equation:
\begin{equation}
\begin{split}
  \left[^4\mbox{He}\right] = &
  8 (e^{\lambda_{238}t} - 1) \mbox{[\textsuperscript{238}U]} + 
  7 (e^{\lambda_{235}t} - 1) \mbox{[\textsuperscript{235}U]} + \\
  ~ & 6 (e^{\lambda_{232}t} - 1) \mbox{[\textsuperscript{232}Th]} +
  (e^{\lambda_{147}t} - 1) \mbox{[\textsuperscript{147}Sm]}
\end{split}
\label{eq:U-Th-He}
\end{equation}

where [\textsuperscript{4}He], [\textsuperscript{238}U],
[\textsuperscript{235}U] [\textsuperscript{232}Th] and
[\textsuperscript{147}Sm] are concentrations in atoms or moles per
unit mass or volume.

Fick's Law:
\begin{equation}
\frac{\partial C}{\partial t} = D \left(
\frac{\partial^2C}{\partial x^2} + \frac{\partial^2C}{\partial y^2} +
\frac{\partial^2C}{\partial z^2}\right)
\label{eq:fick}
\end{equation}

Arrhenius equation:
\begin{equation}
\ln(D) = \ln(D_\circ) - \frac{E_a}{RT}
\label{eq:logD}
\end{equation}

where R = 8.3144621 J/mol.K.\medskip

Closure temperature (assuming that $t \propto 1/T$):
\begin{equation}
\frac{E_a}{RT_c} = \ln\left(\frac{ART_c^2D_\circ/r^2}{E_adT/dt}\right)
\label{eq:Tc}
\end{equation}

where $A$ = 55 for a sphere, 27 for a cylinder and 8.7 for a plane
sheet.

\subsection{Fission tracks}
\label{sec:fission-tracks}

The volume density $n_s$ (in cm$^{-3}$) of fission tracks:
\begin{equation}
n_{s} = \frac{\lambda_f}{\lambda} [^{238}U] \left(e^{\lambda t}-1\right)
\label{eq:Ns}
\end{equation}

The surface density $\rho$ (in cm$^{-2}$):
\begin{equation}
\rho = g L n_s
\label{eq:rhos}
\end{equation}

Where $g=1$ for internal and $g=1/2$ for external surfaces, and
$L\sim$15\textmu{m} for apatite. Age equation:
\begin{equation}
t = \frac{1}{\lambda}
\ln\left(\frac{\lambda}{\lambda_f}\frac{\rho_s}{[^{238}U] g_s L
}+1\right)
\label{eq:tFT}
\end{equation}

External detector method:
\begin{equation}
t =
\frac{1}{\lambda}\ln\left(1+\frac{g_i}{g_s}\lambda\zeta\rho_d\frac{N_s}{N_i}\right)
\label{eq:tzeta}
\end{equation}

\noindent where $N_s$ and $N_i$ are the spontaneous and induced track
counts. Arrhenius relationship for fading fission tracks:
\begin{equation}
\ln(t) = \frac{E_A}{kT} +
\ln\left[\ln\left(\frac{\rho_\circ}{\rho}\right)\right] - \ln(C)
\label{eq:lnt}
\end{equation}

where $k=8.616\times{10}^{-5}$eV/K.

\section[Cosmogenic Nuclides]{Cosmogenic nuclides}
\label{sec:cosmo}

Will not be part of the exam.

\section[U-series dating]{U-series disequilibrium}
\label{ch:intro2Useries}

\subsection{The $^{234}$U-$^{238}$U method}
\label{sec:234238}

Age equation:
\begin{equation}
\frac{A(^{234}U)}{A(^{238}U)} = 1 + [ \gamma_\circ - 1 ] e^{-\lambda_{234}t} 
\label{eq:A234A238}
\end{equation}

\subsection{The $^{230}$Th method}
\label{sec:230}

Ingrowth equation:
\begin{equation}
A(^{230}Th) = A(^{230}Th)_\circ e^{-\lambda_{230}t} + A(^{238}U)(1-e^{-\lambda_{230}t})
\label{eq:A230}
\end{equation}

Isochron equation:
\begin{equation}
  \frac{A(^{230}Th)}{A(^{232}Th)} =
  \frac{A(^{230}Th)_\circ}{A(^{232}Th)} e^{-\lambda_{230}t} + 
  \frac{A(^{238}U)}{A(^{232}Th)}(1-e^{-\lambda_{230}t})
\label{eq:230232}
\end{equation}

\subsection{The $^{230}$Th-U method}
\label{sec:230238}

Decay series:
\begin{align}
~ & A(^{230}Th) =  A(^{230}Th)^s + A(^{230}Th)^x \label{eq:230total}\\
\mbox{with:~} & A(^{230}Th)^s = A(^{238}U) (1-e^{-\lambda_{230}t}) \label{eq:230s}\\
\mbox{and:~} & A(^{230}Th)^x = \frac{\lambda_{230}}{\lambda_{230}-\lambda_{234}} 
A(^{234}U)_\circ^x\left(e^{-\lambda_{234}t}-e^{-\lambda_{230}t}\right) \label{eq:230x}
\end{align}

Age equation:
\begin{equation}
  \frac{A(^{230}Th)}{A(^{238}U)} = 1 - e^{-\lambda_{230}t} +
  \frac{\lambda_{230}}{\lambda_{230}-\lambda_{234}} (\gamma_\circ-1)
\left(e^{-\lambda_{234}t}-e^{-\lambda_{230}t}\right)
\label{eq:230238}
\end{equation}

If $\gamma_\circ = 1$:
\begin{equation}
\frac{A(^{230}Th)}{A(^{238}U)} = 1-e^{-\lambda_{230}t}
\label{eq:230238b}
\end{equation}

\section{Error propagation}
\label{ch:error-propagation}

\subsection{Basic definitions}
\label{sec:summarystatistics}

Mean:
\begin{equation}
\left\{
\begin{array}{rl}
\overline{x} \equiv & \frac{1}{n} \sum_{i=1}^{n} x_i\\
\overline{y} \equiv & \frac{1}{n} \sum_{i=1}^{n} y_i
\end{array}
\right.
\label{eq:mean}
\end{equation}

Variance:
\begin{equation}
\left\{
\begin{array}{rl}
s[x]^2 \equiv & \frac{1}{n-1} \sum_{i=1}^{n} (x_i-\overline{x})^2\\
s[y]^2 \equiv & \frac{1}{n-1} \sum_{i=1}^{n} (y_i-\overline{y})^2
\end{array}
\right.
\label{eq:variance}
\end{equation}

Covariance:
\begin{equation}
s[x,y] \equiv \frac{1}{n-1} \sum_{i=1}^{n} (x_i-\overline{x})(y_i-\overline{y})
\label{eq:covariance}
\end{equation}

General equation for the error propagation of a function $t=f(x,y)$:
\begin{equation}
s[t]^2 s[x]^2 \left(\frac{\partial f}{\partial x}\right)^2 +
s[y]^2 \left(\frac{\partial f}{\partial y}\right)^2 +
2~s[x,y] \frac{\partial f}{\partial x} \frac{\partial f}{\partial y} \label{eq:s2t}
\end{equation}

In matrix form:
\begin{equation}
s[t]^2 = 
\left[
\begin{array}{@{}c@{~}c@{}}
\frac{\partial t}{\partial x}&\frac{\partial t}{\partial y}
\end{array}
\right]
\left[
\begin{array}{@{}c@{~}c@{}}
s[x]^2 & s[x,y]\\
s[x,y] & s[y]^2
\end{array}
\right]
\left[
\begin{array}{@{}c@{}}
\frac{\partial t}{\partial x} \\
\frac{\partial t}{\partial y}
\end{array}
\right]
\label{eq:s2tmatrix}
\end{equation}

\subsection{Examples}

Let $x$ and $y$ indicate measured quantities associated with
analytical uncertainty.  And let $a$ and $b$ be some error free
parameters.
\begin{enumerate}
\item{addition ($t = a x + b y$):}
\begin{equation}
  s[t]^2 = a^2 s[x]^2 + b^2 s[y]^2 + 2ab~s[x,y]
  \label{eq:addition}
\end{equation}

\item{subtraction ($t = a x - b y$):}
\begin{equation}
s[t]^2 = a^2 s[y]^2 + b^2 s[y]^2 - 2ab~s[x,y]
\label{eq:subtraction}
\end{equation}

\item{multiplication ($t = a x y$):}
\begin{equation}
\left(\frac{s[t]}{t}\right)^2 = \left(\frac{s[x]}{x}\right)^2 + 
  \left(\frac{s[y]}{y}\right)^2 + 2 \frac{s[x,y]}{x y}
\label{eq:multiplication}
\end{equation}

\item{division ($t = a \frac{x}{y}$):}
\begin{equation}
  \left(\frac{s[t]}{t}\right)^2 = \left(\frac{s[x]}{x}\right)^2 + 
  \left(\frac{s[y]}{y}\right)^2 - 2 \frac{s[x,y]}{x y}
\label{eq:division}
\end{equation}

\item{exponentiation ($t = a~e^{bx}$):}
\begin{equation}
s[t]^2 = (b t)^2 s[x]^2
\label{eq:exponentiation}
\end{equation}

\item{logarithms ($t = a~\ln(bx)$):}
\begin{equation}
s[t]^2 = a^2 \left(\frac{s[x]}{x}\right)^2
\label{eq:logarithms}
\end{equation}

\item{power ($t = a x^b$):}
\begin{equation}
\left(\frac{s[t]}{t}\right)^2 = b^2\left(\frac{s[x]}{x}\right)^2
\label{eq:power}
\end{equation}

\end{enumerate}

\subsection{Standard error of the mean}

If $cov(x_i,x_j)=0 ~\forall~ i, j$:
\begin{equation}
s[\overline{x}]^2 = \frac{1}{n} \sum_{i=1}^{n} s[x_i]^2 =
\frac{s[x]^2}{n} \Rightarrow s[\overline{x}] = \frac{s[x]}{\sqrt{n}}
\label{eq:varianceofthemean}
\end{equation}

\subsection{Poissonian counting statistics}

\begin{equation}
  s[N]^2 = N
\end{equation}

\end{multicols}

\end{document}
