\section{Error propagation}
\label{sec:errorprop-R}

This exercise will build on the results from the previous two
practicals.

\begin{enumerate}

\item Plot the $^{206}$Pb/$^{238}$U-ratios of the sample against those
  of the standard (data from Section \ref{sec:U-Pb-R}).  Verify that
  the covariance between the two can safely be neglected.

\item Calculate the standard errors of the mean $^{206}$Pb/$^{238}$U
  signal ratios for the sample (91500) and the standard (GJ-1) using
  the \texttt{mean} and \texttt{sd} \ifuclnotes (for \texttt{R}) or \texttt{std}
  (for \texttt{Matlab}) \fi functions.

\item Propagate the standard errors of the atomic
  \textsuperscript{206}Pb/\textsuperscript{238}U and
  \textsuperscript{207}Pb/\textsuperscript{235}U-ratios calculated in
  step~\ref{it:atomicUPb} of Section~\ref{sec:U-Pb-R}.

\item Propagate the analytical uncertainties of the U-Pb age, ignoring
  the covariance terms. Recall that 

$$ t = \frac{1}{\lambda_{238}}\ln\left(1 +
  \frac{{}^{206}Pb}{{}^{238}U}\right) $$

If you want you can use the simplifying approximation that $\ln(1+X)
\approx X$ if $X \ll 1$ (this assumption may not be correct for the
$^{207}$Pb/$^{235}$U-age). \label{loglin}

\item Compute the analytical uncertainties associated with the linear
  extrapolation of the argon signals of the sample and the standard in
  Section \ref{sec:Ar-Ar-R}. In \texttt{R}, the covariance matrix of
  the slope and intercept can be simply obtained from the
  \texttt{vcov(fit)} function, where \texttt{fit} is the output of the
  \texttt{lm} function (see item~\ref{itm:lm} of
  Section~\ref{sec:R}). The corresponding standard errors are then
  found by taking the square root of the diagonal elements of this
  matrix.  \ifuclnotes For \texttt{Matlab}, all the relevant
  information is provided in the output of the \texttt{polyfit}
  function:

\begin{console}
help(polyfit)
...
[P,S] = POLYFIT(X,Y,N) returns the polynomial
coefficients P and a structure S for use with 
POLYVAL to obtain error estimates for predictions.  
S contains fields for the triangular factor (R)
from a QR decomposition of the Vandermonde matrix 
of X, the degrees of freedom (df), and the norm 
of the residuals (normr). If the data Y are 
random, an estimate of the covariance matrix 
of P is (Rinv*Rinv')*normr^2/df, where Rinv 
is the inverse of R.
\end{console}

Based on these instructions, I have written the following function to
calculate the standard errors of the fitting parameters from the
second output parameter of the \texttt{polyfit} function:

\begin{script}
function se = S2se(S)
  covmat = (inv(S.R)*inv(S.R'))*S.normr^2/S.df;
  se = sqrt(diag(covmat));
end
\end{script}
\fi

\item Use these error estimates to propagate the analytical
  uncertainty of the J-value and the sample age. Again you can use the
  linear approximation to the age equation mentioned in point
  \ref{loglin}.

\end{enumerate}
