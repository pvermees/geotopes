\section{Introduction to \texttt{R}}
\label{sec:R}

A number of different graphical user interfaces (GUIs) are available
to interact with \texttt{R}. The most popular of these are
\texttt{Rgui}, \texttt{RStudio}, \texttt{RCommander} and
\texttt{Tinn-R}.

\begin{enumerate}
  \item Starting these applications or running \texttt{R} in a text
    terminal presents the user with a command line prompt. Anything
    that is typed after the \texttt{>} symbol will be evaluated
    immediately. Thus, we can use \texttt{R} as a calculator:

\begin{console}
> 1 + 1
> sqrt(2)
> exp(log(10))
> 31 %% 14
\end{console}

\item Alternatively, code can also be saved as text (using a built-in text
  editor) and saved as \texttt{mycode.R}, say. This code can then be
  copied and pasted at the command line prompt. Or it can be called from
  the command line using the \texttt{source()} function:

\begin{console}
> source('mycode.R')
\end{console}

\item In the remainder of this tutorial, we will assume that your code is
  run from a text file unless explicitly stated otherwise. The
  \texttt{\#}~symbol marks the beginning of a comment.  \texttt{R}
  ignores anything that follows it:

\begin{script}
# The arrow symbol is used to assign a value to a
# variable. Note that the arrow can point both ways:
foo <- 2
4 -> bar
print(foo*bar)

# Defining a vector of multiple values:
myvec <- c(0,1,2,3,4,5)
# or, equivalently:
myvec <- seq(0,5)
# or
myvec <- seq(from=0,to=5,by=1)
# or
myvec <- 0:5

# Turn myvec into a 2x3 matrix:
mymat <- matrix(myvec,nrow=2)

# Accessing one or more items from a vector or matrix:
x <- myvec[3]
y <- myvec[1:3]
z <- mymat[1,2:3]

# Change the 2nd value in mymat's 3rd column to 10:
mymat[2,3] <- 10

# Change the entire second column of mymat to -1:
mymat[,2] <- -1

# Transpose of a matrix:
flipped <- t(mymat)

# Element-wise multiplication (*) 
# vs. matrix multiplication (%*%):
rectangle <- mymat * mymat
square <- mymat %*% flipped
\end{script}

\item Lists are used to store more complex data objects:

\begin{script}
mylist <- list(v=myvec,m=mymat,nine=9)
\end{script}

The three components of \texttt{mylist} can be accessed in a number of
equivalent ways. For example, the first item (\texttt{v}) of
\texttt{mylist} can be accessed either as \verb|mylist$v|, as
\verb|mylist[[1]]| or as \verb|mylist[['v']]|.\\

Data frames are list-like tables:

\begin{script}
myframe <- data.frame(period=c('Cz','Mz','Pz','PC'),
                      SrSr=c(0.708,0.707,0.709,0.708),
                      fossils=c(TRUE,TRUE,TRUE,FALSE))
\end{script}

You can access the items in the data frame \texttt{myframe} either
like a list (e.g. \verb|myframe$period|) or like a matrix
(\verb|myframe[,'period']|):

\item If you want to learn more about a function, type \texttt{help()} or
\texttt{?} at the command line prompt:

\begin{console}
> help(c)
> ?matrix
\end{console}

\item In addition to \texttt{R}'s built-in functions, you can also define
  your own:

\begin{script}
cube <- function(n){
    return(n^3)
}

# Using this function to take the cube of 3:
c3 <- cube(3)

# Conditional statement:
toss <- function(){
    if (runif(1)>0.5){ # runif(n) draws n random 
        print("head")  # numbers between 0 and 1
    } else {
        print("tail")
    }
}

# Use a for loop to toss 10 virtual coins:
for (i in 1:10) {
    toss()
}
\end{script}

\item The purpose of the practical exercises in
  Sections~\ref{sec:U-Pb-R}-\ref{sec:FT-R} is to process datasets
  contained in external data files. For this you will need to be able
  to navigate through your file system and load the necessary files:

\begin{console}
> ls()          # list all the variables
> rm(list=ls()) # clear the current workspace
> getwd()       # get the current working directory
> setwd("path_to_a_valid_directory")
\end{console}

\item Use the above commands to navigate to the directory containing the
file named \texttt{RbSr.csv}. Then read this file into memory:

\begin{script}
RbSr <- read.csv("RbSr.csv",header=TRUE)
\end{script}

Type \texttt{names(RbSr)} or \texttt{colnames(RbSr)} at the command
prompt to list the variable names (column headers) contained in this
dataset.

\item Let us now perform an isochron regression
  (Section~\ref{sec:Rb-Sr}) through these Rb-Sr data: \label{itm:lm}

\begin{script}[firstnumber=2]
# Plot Sr87Sr86 against Rb87Sr86:
plot(RbSr[,'Rb87Sr86'],RbSr[,'Sr87Sr86'],type="p")

# fit a linear model to the data
fit <- lm(Sr87Sr86 ~ Rb87Sr86, data = RbSr)
\end{script}

\item \texttt{fit} is a `list' object. Type \texttt{str(fit)} at the
  command prompt to see its structure. One of its items is
  \verb|fit$coefficients|, which contains the slope and the intercept
  of the linear fit. Alternatively, we can also access these values
  using the \texttt{coef()} function. The following code uses this
  function to calculate the isochron age:

\begin{script}[firstnumber=7]
# define the 87Rb decay constant (in Ma-1):
lam87 <- 1.3972e-5 # according to Villa et al. (2015)
# compute the age from the slope:
tRbSr <- log(1 + coef(fit)[2])/lam87

# add the best fit line to the existing plot:
abline(fit)
# label with the isochron age: 
title(tRbSr)
\end{script}

\item\label{it:installingIsoplotR} One of the most powerful features
  of \texttt{R} is the availability of thousands of `packages'
  providing additional functionality to the built-in functions.  For
  example, let us download and install the \texttt{IsoplotR} package
  from the `Comprehensive R Archive Network' (CRAN):

\begin{console}
> install.packages('IsoplotR')
\end{console}

\item We can use \texttt{IsoplotR} to redo the isochron regression
  exercise using a more rigorous weighted regression algorithm that
  takes into account the analytical uncertainties in both the
  \textsuperscript{87}Sr/\textsuperscript{86}Sr- and
  \textsuperscript{87}Rb/\textsuperscript{86}Sr-ratios:

\begin{script}
# load the functionality of the IsoplotR package:
library('IsoplotR')

# load the data (see ?read.data for details):
RbSr2 <- read.data('RbSr.csv',method='Rb-Sr',format=1)

# compute and plot the isochron diagram:
isochron(RbSr2)
\end{script}

Even though \texttt{IsoplotR} is powerful, convenient, and popular, we
will not use it in the remainder of these notes. Instead, we will
carry out all our calculations in base \texttt{R} because this will
give you a more fundamental understanding of geochronology.
\end{enumerate}
